\section{KLEE}

\subsection{Инсталляция}

Процесс сборки KLEE из исходников сложный и запутанный.
Самый простой способ использовать KLEE, это инсталлировать docker
\footnote{\url{https://docs.docker.com/engine/installation/linux/ubuntulinux/}} а затем запустить образ KLEE для docker
\footnote{\url{http://klee.github.io/docker/}}.
Путь, где находятся файлы KLEE, может выглядет так:
\textbf{/var/lib/docker/aufs/mnt/(много шестнадцетиричных цифр)/home/klee}.

% subsections:
\subsection{Головоломка зебры (\ac{AKA} Загадка Эйнштейна)}
\label{zebra_SMT}

Головоломка зебры это популярная головоломка, определяется так:

% FIXME remove paragraph at first line
\begin{framed}
\begin{quotation}
1. На улице стоят пять домов.\\
2. Англичанин живёт в красном доме.\\
3. У испанца есть собака.\\
4. В зелёном доме пьют кофе.\\
5. Украинец пьёт чай.\\
6. Зелёный дом стоит сразу справа от белого дома.\\
7. Тот, кто курит Old Gold, разводит улиток.\\
8. В жёлтом доме курят Kool.\\
9. В центральном доме пьют молоко.\\
10. Норвежец живёт в первом доме.\\
11. Сосед того, кто курит Chesterfield, держит лису.\\
12. В доме по соседству с тем, в котором держат лошадь, курят Kool.\\
13. Тот, кто курит Lucky Strike, пьёт апельсиновый сок.\\
14. Японец курит Parliament.\\
15. Норвежец живёт рядом с синим домом.\\
\\
Кто пьёт воду? Кто держит зебру?\\
\\
В целях ясности следует добавить, что каждый из пяти домов окрашен в свой цвет, а их жители — разных национальностей, владеют разными животными, пьют разные напитки и курят разные марки американских сигарет. Ещё одно замечание: в утверждении 6 справа означает справа относительно вас.
\end{quotation}
\end{framed}
( \url{http://bit.ly/2oUNBhc} (Wikipedia) ) \\
\\
Это очень хороший пример \ac{CSP}.

Мы можем закодировать каждый объект как целочисленную переменную, определяющую номер дома.

Тогда, чтобы определить, что англичанин живет в красном доме, мы добавим этот констрайнт: \TT{Englishman == Red}, означающий, что номер дома, где живет англичанин, и номер дома окрашенный в красный --- один и тот же.

Мы определяем что норвежец живет в соседнем доме с синим домом, но мы точно не знаем, слева от синего дома, или справа,
но мы знаем что номер дома будет отличается на 1.
Так что определим такой констрайнт: \TT{Norwegian==Blue-1 OR Norwegian==Blue+1}.

Также нужно ограничить номера всех домов, чтобы они были в пределах 1..5.

Мы также будем использовать \TT{Distinct}, чтобы показать, что все различные объекты одного типа должны находиться в домах
с разными номерами.

\lstinputlisting[style=custompy]{SMT/zebra.py}

Запускаем и получаем корректный результат:

\begin{lstlisting}
sat
[Snails = 3,
 Blue = 2,
 Ivory = 4,
 OrangeJuice = 4,
 Parliament = 5,
 Yellow = 1,
 Fox = 1,
 Zebra = 5,
 Horse = 2,
 Dog = 4,
 Tea = 2,
 Water = 1,
 Chesterfield = 2,
 Red = 3,
 Japanese = 5,
 LuckyStrike = 4,
 Norwegian = 1,
 Milk = 3,
 Kools = 1,
 OldGold = 3,
 Ukrainian = 2,
 Coffee = 5,
 Green = 5,
 Spaniard = 4,
 Englishman = 3]
\end{lstlisting}


\input{KLEE/color_RU.tex}
\input{KLEE/strcmp_RU.tex}
\subsection{Дата и время в UNIX}

Дата и время в UNIX\footnote{\url{https://en.wikipedia.org/wiki/Unix_time}} это число секунд прошедших с
1-Jan-1970 00:00 UTC.
Ф-ция gmtime() в Си/Си++ используется для декодирования этого значения в строку, понятную человеку.

Вот фрагмент кода, который я скопипастил из древней версии ОС Minix:
(\url{http://www.cise.ufl.edu/~cop4600/cgi-bin/lxr/http/source.cgi/lib/ansi/gmtime.c}) и немного переработал:

\lstinputlisting[numbers=left]{KLEE/klee_time1.c}

Попробуем:

\begin{lstlisting}
% clang -emit-llvm -c -g klee_time1.c
...

% klee klee_time1.bc
KLEE: output directory is "/home/klee/klee-out-107"
KLEE: WARNING: undefined reference to function: printf
KLEE: ERROR: /home/klee/klee_time1.c:86: external call with symbolic argument: printf
KLEE: NOTE: now ignoring this error at this location
KLEE: ERROR: /home/klee/klee_time1.c:83: ASSERTION FAIL: 0
KLEE: NOTE: now ignoring this error at this location

KLEE: done: total instructions = 101579
KLEE: done: completed paths = 1635
KLEE: done: generated tests = 2
\end{lstlisting}

Ух ты, на строке 83 сработал assert(), почему?
Посмотрим, какое значение UNIX-времени привело к этому:

\begin{lstlisting}
% ls klee-last | grep err
test000001.exec.err
test000002.assert.err

% ktest-tool --write-ints klee-last/test000002.ktest
ktest file : 'klee-last/test000002.ktest'
args       : ['klee_time1.bc']
num objects: 1
object    0: name: b'time'
object    0: size: 4
object    0: data: 978278400
\end{lstlisting}

Попробуем декодировать это значение используя утилиту date в UNIX:

\begin{lstlisting}
% date -u --date='@978278400'
Sun Dec 31 16:00:00 UTC 2000
\end{lstlisting}

После изучения, я нашел что переменная \TT{month} может содержать неверное значение 12 (для которого максимальное это 11,
для декабря), 
потому что макрос LEAPYEAR() должен принимать на вход год как 2000, а не как 100.
Так что пока я переписывал ф-цию, я сделал ошибку, и KLEE нашла её!

Просто интересно, что будет если я заменю switch() на массив строк, как это обычно пишется в кратком коде на Си/Си++?

\begin{lstlisting}
	...

const char *_months[] =
{
	"January", "February", "March",
	"April", "May", "June",
	"July", "August", "September",
	"October", "November", "December"
};

	...

	while (dayno >= _ytab[LEAPYEAR(year)][month])
	{
		dayno -= _ytab[LEAPYEAR(year)][month];
		month++;
	}
	
	char *s=_months[month];

	printf ("%04d-%s-%02d %02d:%02d:%02d\n", YEAR0+year, s, dayno+1, hour, minutes, seconds);
	printf ("week day: %s\n", _days[wday]);	
	
	...

\end{lstlisting}

KLEE обнаруживает попытку прочитать за границами массива:

\begin{lstlisting}
% klee klee_time2.bc
KLEE: output directory is "/home/klee/klee-out-108"
KLEE: WARNING: undefined reference to function: printf
KLEE: ERROR: /home/klee/klee_time2.c:69: external call with symbolic argument: printf
KLEE: NOTE: now ignoring this error at this location
KLEE: ERROR: /home/klee/klee_time2.c:67: memory error: out of bound pointer
KLEE: NOTE: now ignoring this error at this location

KLEE: done: total instructions = 101716
KLEE: done: completed paths = 1635
KLEE: done: generated tests = 2
\end{lstlisting}

Это то же самое UNIX-время, которое мы уже видели:

\begin{lstlisting}
% ls klee-last | grep err
test000001.exec.err
test000002.ptr.err

% ktest-tool --write-ints klee-last/test000002.ktest
ktest file : 'klee-last/test000002.ktest'
args       : ['klee_time2.bc']
num objects: 1
object    0: name: b'time'
object    0: size: 4
object    0: data: 978278400
\end{lstlisting}

Так что, если этот фрагмент кода может быть выполнен на удаленном компьютере, с этим входным значением
(\textit{input of death}),
Так можно свалить процесс (хотя и с какой-то удачей).\\
\\
ОК, теперь я исправляю ошибку, перемещая выражение где отнимается год на строку 43, и теперь посмотрим,
какое UNIX-время соответствует некоторой красивой дате вроде 2022-February-2?

\lstinputlisting[numbers=left]{KLEE/klee_time3.c}

\begin{lstlisting}
% clang -emit-llvm -c -g klee_time3.c
...

% klee klee_time3.bc
KLEE: output directory is "/home/klee/klee-out-109"
KLEE: WARNING: undefined reference to function: klee_assert
KLEE: WARNING ONCE: calling external: klee_assert(0)
KLEE: ERROR: /home/klee/klee_time3.c:47: failed external call: klee_assert
KLEE: NOTE: now ignoring this error at this location

KLEE: done: total instructions = 101087
KLEE: done: completed paths = 1635
KLEE: done: generated tests = 1635

% ls klee-last | grep err
test000587.external.err

% ktest-tool --write-ints klee-last/test000587.ktest
ktest file : 'klee-last/test000587.ktest'
args       : ['klee_time3.bc']
num objects: 1
object    0: name: b'time'
object    0: size: 4
object    0: data: 1645488640

% date -u --date='@1645488640'
Tue Feb 22 00:10:40 UTC 2022
\end{lstlisting}

Успешно нашли, но часы/минуты/секунды выглядят как случайные --- они и правда случайные, потому что KLEE удовлетворило
все констрайнты нами добавленные, ничего более.
Мы ведь не просили выставить часы/минуты/секунды в нули.

Добавим также констрайнты для часом/минут/секунд:

\begin{lstlisting}
	...

	if (YEAR0+year==2022 && month==1 && dayno+1==22 && hour==22 && minutes==22 && seconds==22)
		klee_assert(0);
	
	...
\end{lstlisting}

Запустим и проверим \dots

\begin{lstlisting}
% ktest-tool --write-ints klee-last/test000597.ktest
ktest file : 'klee-last/test000597.ktest'
args       : ['klee_time3.bc']
num objects: 1
object    0: name: b'time'
object    0: size: 4
object    0: data: 1645568542

% date -u --date='@1645568542'
Tue Feb 22 22:22:22 UTC 2022
\end{lstlisting}

Теперь всё точно.

Да, конечно, в библиотеках Си/Си++ есть ф-ции для кодирования строки с датой в UNIX-время, но то что мы тут получили,
это KLEE работающий как \textit{антипод} декодирующей ф-ции, \textit{инверсная ф-ция} в каком-то смысле.

\input{KLEE/base64_RU.tex}
\input{KLEE/LZSS_RU.tex}
\subsection{strtodx() из RetroBSD}

Нашел эту ф-цию в RetroBSD:
\url{https://github.com/RetroBSD/retrobsd/blob/master/src/libc/stdlib/strtod.c}.
Она конвертирует строку в число с плавающей точкой для заданной системы исчисления.

\lstinputlisting[numbers=left]{KLEE/strtodx.c}
( \url{https://github.com/DennisYurichev/SAT_SMT_article/blob/master/KLEE/strtodx.c} )

Интересно, KLEE не поддерживает арифметику с плавающей точкой, но тем не менее, что-то нашел:

\begin{lstlisting}
...

KLEE: ERROR: /home/klee/klee_test.c:202: memory error: out of bound pointer

...

% ktest-tool klee-last/test003483.ktest
ktest file : 'klee-last/test003483.ktest'
args       : ['klee_test.bc']
num objects: 1
object    0: name: b'buf'
object    0: size: 10
object    0: data: b'-.0E-66\x00\x00\x00'
\end{lstlisting}

Как видно, строка ``-.0E-66'' приводит к чтению из массива за его пределами, в строке 202.
Во время дальнейшего изучения, я обнаружил что массив \TT{powersOf10[]} слишком короткий:
был прочитан 6-й элемент (считая с нулевого).
И мы видим что часть массива закомментирована (строка 79)!
Вероятно, чья-то ошибка?


\subsection{Unit-тестирование: простой калькулятор}

Искал простой калькулятор, который принимает на вход выражение вроде ``2+2'' и выдает ответ.
Нашел один здесь: \url{http://stackoverflow.com/a/13895198}.
К сожалению, в нем не было ошибок, так что я добавил одну: буфер токенов (\TT{buf[]} на строке) короче чем входной буфер (\TT{input[]} на строке).

\lstinputlisting[numbers=left]{KLEE/calc.c}
( \url{https://github.com/DennisYurichev/SAT_SMT_article/blob/master/KLEE/calc.c} )

KLEE легко нашел переполнение буфера (65 нулей + один символ табуляции):

\begin{lstlisting}
% ktest-tool --write-ints klee-last/test000468.ktest
ktest file : 'klee-last/test000468.ktest'
args       : ['calc.bc']
num objects: 1
object    0: name: b'input'
object    0: size: 128
object    0: data: b'0\t0000000000000000000000000000000000000000000000000000000000000000\xff\xff\xff\xff\xff\xff\xff\xff\xff\xff\xff\xff\xff\xff\xff\xff\xff\xff\xff\xff\xff\xff\xff\xff\xff\xff\xff\xff\xff\xff\xff\xff\xff\xff\xff\xff\xff\xff\xff\xff\xff\xff\xff\xff\xff\xff\xff\xff\xff\xff\xff\xff\xff\xff\xff\xff\xff\xff\xff\xff\xff\xff'
\end{lstlisting}

Трудно сказать, как в массив input[] попал символ табуляции (\TT{\textbackslash{}t})? но KLEE достиг желаемого: буфер переполнился.\\
\\
KLEE также нашла две строки с выражениями, которые приводят к делению на ноль (``0/0'' и ``0\%0''):

\begin{lstlisting}
% ktest-tool --write-ints klee-last/test000326.ktest
ktest file : 'klee-last/test000326.ktest'
args       : ['calc.bc']
num objects: 1
object    0: name: b'input'
object    0: size: 128
object    0: data: b'0/0\x00\xff\xff\xff\xff\xff\xff\xff\xff\xff\xff\xff\xff\xff\xff\xff\xff\xff\xff\xff\xff\xff\xff\xff\xff\xff\xff\xff\xff\xff\xff\xff\xff\xff\xff\xff\xff\xff\xff\xff\xff\xff\xff\xff\xff\xff\xff\xff\xff\xff\xff\xff\xff\xff\xff\xff\xff\xff\xff\xff\xff\xff\xff\xff\xff\xff\xff\xff\xff\xff\xff\xff\xff\xff\xff\xff\xff\xff\xff\xff\xff\xff\xff\xff\xff\xff\xff\xff\xff\xff\xff\xff\xff\xff\xff\xff\xff\xff\xff\xff\xff\xff\xff\xff\xff\xff\xff\xff\xff\xff\xff\xff\xff\xff\xff\xff\xff\xff\xff\xff\xff\xff\xff\xff\xff'

% ktest-tool --write-ints klee-last/test000557.ktest
ktest file : 'klee-last/test000557.ktest'
args       : ['calc.bc']
num objects: 1
object    0: name: b'input'
object    0: size: 128
object    0: data: b'0%0\x00\xff\xff\xff\xff\xff\xff\xff\xff\xff\xff\xff\xff\xff\xff\xff\xff\xff\xff\xff\xff\xff\xff\xff\xff\xff\xff\xff\xff\xff\xff\xff\xff\xff\xff\xff\xff\xff\xff\xff\xff\xff\xff\xff\xff\xff\xff\xff\xff\xff\xff\xff\xff\xff\xff\xff\xff\xff\xff\xff\xff\xff\xff\xff\xff\xff\xff\xff\xff\xff\xff\xff\xff\xff\xff\xff\xff\xff\xff\xff\xff\xff\xff\xff\xff\xff\xff\xff\xff\xff\xff\xff\xff\xff\xff\xff\xff\xff\xff\xff\xff\xff\xff\xff\xff\xff\xff\xff\xff\xff\xff\xff\xff\xff\xff\xff\xff\xff\xff\xff\xff\xff\xff\xff\xff'
\end{lstlisting}

Может это и не впечатляющий результат, тем не менее, это еще одно напоминание что операции деления и вычисления остатка должны быть обернуты как-то в продакшене, чтобы избежать возможного падения.


\input{KLEE/regexp_RU.tex}

\subsection{Еще примеры}

\url{https://feliam.wordpress.com/2010/10/07/the-symbolic-maze/}

\subsection{Упражнение}

Вот мой crackme/keygenme, который может быть очень запутанным, но его очень легко решить используя KLEE:
\url{http://challenges.re/74/}.


