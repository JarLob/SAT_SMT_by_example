\subsection{Can rand() generate 10 consecutive zeroes?}

\renewcommand{\CURPATH}{equations/LCG}

I've always been wondering, if it's possible or not.
As of simplest linear congruential generator from MSVC's rand(), I could get a state at which rand() will output 8 zeroes modulo 10:

\lstinputlisting{\CURPATH/LCG10.py}

\begin{lstlisting}
sat
[state3 = 1181667981,
 state4 = 342792988,
 state5 = 4116856175,
 state7 = 1741999969,
 state8 = 3185636512,
 state2 = 1478548498,
 state6 = 4036911734,
 state1 = 286227003,
 state9 = 1700675811]
\end{lstlisting}

This is a case if, in some video game, you'll find a code:

\begin{lstlisting}
for (int i=0; i<8; i++)
    printf ("%d\n", rand() % 10);
\end{lstlisting}

... and at some point, this piece of code can generate 8 zeroes in row, if the state will be 286227003 (decimal).

Just checked this piece of code in MSVC 2015:

\begin{lstlisting}
// MSVC 2015 x86

#include <stdio.h>

int main()
{
	srand(286227003);

	for (int i=0; i<8; i++)
		printf ("%d\n", rand() % 10);
};
\end{lstlisting}

Yes, its output is 8 zeroes!

What about other modulos?

I can get 4 consecutive zeroes modulo 100:

\lstinputlisting{\CURPATH/LCG100.py}

\begin{lstlisting}
sat
[state3 = 635704497,
 state4 = 1644979376,
 state2 = 1055176198,
 state1 = 3865742399,
 state5 = 1389375667]
\end{lstlisting}

However, 4 consecutive zeroes modulo 100 is impossible (given these constants at least), this gives ``unsat'':
\url{https://github.com/DennisYurichev/yurichev.com/...LCG100_v1.py}.

... and 3 consecutive zeroes modulo 1000:

\lstinputlisting{\CURPATH/LCG1000.py}

\begin{lstlisting}
sat
[state3 = 1179663182,
 state2 = 720934183,
 state1 = 4090229556,
 state4 = 786474201]
\end{lstlisting}

What if we could use rand()'s output without division? Which is in 0..0x7fff range (i.e., 15 bits)?
As it can be checked quickly, 2 zeroes at output is possible:

\lstinputlisting{\CURPATH/LCG.py}

\begin{lstlisting}
sat
[state2 = 20057, state1 = 3385131726, state3 = 22456]
\end{lstlisting}

\subsubsection{UNIX time and srand(time(NULL))}

Given the fact that it's highly popular to initialize LCG PRNG with UNIX time (i.e., \TT{srand(time(NULL))}), you can probably calculate a moment in time so that LCG PRNG will be initialized as you want to.

For example, can we get a moment in time from now (5-Dec-2017) till 12-Dec-2017 (that is one week from now), when, if initialized by UNIX time, rand() will output as many similar numbers (modulo 10), as possible?

\lstinputlisting{\CURPATH/LCG10_time.py}

Yes:

\begin{lstlisting}
sat
[state3 = 2234253076,
 state4 = 497021319,
 state5 = 4160988718,
 c = 3,
 state2 = 333151205,
 state6 = 46785593,
 state1 = 1512500810,
 state7 = 1158878744]
\end{lstlisting}

If \TT{srand(time(NULL))} will be executed at \TT{Tue Dec  5 21:06:50 EET 2017} (this precise second, UNIX time=1512500810), a next 6 \TT{rand() \% 10} lines will output six numbers of 3 in a row.
Don't know if it useful or not, but you've got the idea.

\subsubsection{etc:}

The files: \url{https://github.com/DennisYurichev/yurichev.com/...}

Further work: check glibc's rand(), Mersenne Twister, etc. Simple 32-bit LCG as described can be checked using simple brute-force, I think.

