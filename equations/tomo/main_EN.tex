\subsection{Discrete tomography}

How computed tomography (CT scan) actually works?
A human body is bombarded by X-rays in various angles by X-ray tube in rotating torus.
X-ray detectors are also located in torus, and all the information is recorded.

Here is we can simulate simple tomograph.
An ``i'' character is rotating and will be ``enlighten'' at 4 angles.
Let's imagine, character is bombarded by X-ray tube at left.
All asterisks in each row is then summed and sum is "received" by X-ray detector at the right.

\begin{lstlisting}
WIDTH= 11 HEIGHT= 11
angle=(π/4)*0
    **      2
    **      2
            0
   ***      3
    **      2
    **      2
    **      2
    **      2
    **      2
   ****     4
            0
[2, 2, 0, 3, 2, 2, 2, 2, 2, 4, 0] ,
angle=(π/4)*1
            0
            0
  *         1
 **         2
    *       1
    **      2
     **     2
     ****   4
       *    1
      *     1
            0
[0, 0, 1, 2, 1, 2, 2, 4, 1, 1, 0] ,
angle=(π/4)*2
            0
            0
            0
            0
         *  1
** *******  9
** *******  9
   *     *  2
            0
            0
            0
[0, 0, 0, 0, 1, 9, 9, 2, 0, 0, 0] ,
angle=(π/4)*3
            0
            0
       *    1
       **   2
      ** *  3
     ***    3
    **      2
            0
  **        2
   *        1
            0
[0, 0, 1, 2, 3, 3, 2, 0, 2, 1, 0] ,
\end{lstlisting}

( The source code: \url{https://github.com/DennisYurichev/SAT_SMT_by_example/blob/master/equations/tomo/gen.py} )

All we got from our toy-level tomograph is 4 vectors, these are sums of all asterisks in rows for 4 angles:

\begin{lstlisting}
[2, 2, 0, 3, 2, 2, 2, 2, 2, 4, 0] ,
[0, 0, 1, 2, 1, 2, 2, 4, 1, 1, 0] ,
[0, 0, 0, 0, 1, 9, 9, 2, 0, 0, 0] ,
[0, 0, 1, 2, 3, 3, 2, 0, 2, 1, 0] ,
\end{lstlisting}

How do we recover initial image?
We are going to represent 11*11 matrix, where sum of each row must be equal to some value we already know.
Then we rotate matrix, and do this again.

For the first matrix, the system of equations looks like that (we put there a values from the first vector):

\begin{lstlisting}
C1,1 + C1,2 + C1,3 + ... + C1,11 =      2
C2,1 + C2,2 + C2,3 + ... + C2,11 =      2

...

C10,1 + C10,2 + C10,3 + ... + C10,11 =  4
C11,1 + C11,2 + C11,3 + ... + C11,11 =  0
\end{lstlisting}

We also build similar systems of equations for each angle.

The ``rotate'' function has been taken from the generation program, because, due to Python's dynamic typization nature,
it's not important for the function to what operate on:
strings, characters, or Z3 variable instances, so it works very well for all of them.

\lstinputlisting[style=custompy]{equations/tomo/solve.py}

( The source code: \url{https://github.com/DennisYurichev/SAT_SMT_by_example/blob/master/equations/tomo/solve.py} )

That works:

\begin{lstlisting}
% python solve.py
sat
    **
    **

   ***
    **
    **
    **
    **
    **
   ****
\end{lstlisting}

In other words, all SMT-solver does here is solving a system of equations.

So, 4 angles are enough.
What if we could use only 3 angles?

\begin{lstlisting}
WIDTH= 11 HEIGHT= 11
angle=(π/3)*0
    **      2
    **      2
            0
   ***      3
    **      2
    **      2
    **      2
    **      2
    **      2
   ****     4
            0
[2, 2, 0, 3, 2, 2, 2, 2, 2, 4, 0] ,
angle=(π/3)*1
            0
            0
            0
 **         2
 **         2
   ***      3
     ****   4
       **   2
       *    1
            0
            0
[0, 0, 0, 2, 2, 3, 4, 2, 1, 0, 0] ,
angle=(π/3)*2
            0
            0
            0
       **   2
       **   2
     *****  5
    **      2
 **         2
  *         1
            0
            0
[0, 0, 0, 2, 2, 5, 2, 2, 1, 0, 0] ,
\end{lstlisting}

No, it's not enough:

\begin{lstlisting}
% time python solve3.py
sat
 *  *
    **

     * **
   **
   *  *
    **
     *   *
*   *
   ****
\end{lstlisting}

However, the result is correct, but only 3 vectors allows too many possible ``initial images'',
and Z3 SMT-solver finds first.

Further reading:
\url{https://en.wikipedia.org/wiki/Discrete_tomography},
\url{https://en.wikipedia.org/wiki/2-satisfiability#Discrete_tomography}.

