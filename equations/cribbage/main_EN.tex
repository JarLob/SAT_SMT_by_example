\subsection{Cribbage}

I've found this problem in the Ronald L. Graham, Donald E. Knuth, Oren Patashnik -- ``Concrete Mathematics'' book:

\begin{framed}
\begin{quotation}
Cribbage players have long been aware that 15 = 7 + 8 = 4 + 5 + 6 =
1 + 2 + 3 + 4 + 5 . Find the number of ways to represent 1050 as a sum of
consecutive positive integers. (The trivial representation `1050' by itself
counts as one way; thus there are four, not three, ways to represent 15
as a sum of consecutive positive integers. Incidentally, a knowledge of
cribbage rules is of no use in this problem.)
\end{quotation}
\end{framed}

My solution:

\lstinputlisting[style=custompy]{equations/cribbage/cribbage.py}

The result:

\begin{lstlisting}
(3 terms) 349 + ... + 351 == 1050
(4 terms) 261 + ... + 264 == 1050
(5 terms) 208 + ... + 212 == 1050
(7 terms) 147 + ... + 153 == 1050
(12 terms) 82 + ... + 93 == 1050
(15 terms) 63 + ... + 77 == 1050
(20 terms) 43 + ... + 62 == 1050
(21 terms) 40 + ... + 60 == 1050
(25 terms) 30 + ... + 54 == 1050
(28 terms) 24 + ... + 51 == 1050
(35 terms) 13 + ... + 47 == 1050
\end{lstlisting}

