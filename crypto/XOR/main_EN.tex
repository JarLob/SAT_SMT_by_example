\subsection{Cracking simple XOR cipher with Z3}
\label{XOR_Z3}

\renewcommand{\CURPATH}{crypto/XOR}

Here is a problem: a text encrypted with simple XOR cipher.
Trying all possible keys is not an option.

Relationships between plain text, cipher text and key can be described using simple system of equations.
But we can do more: we can ask Z3 to find such a key (array of bytes), so the plain text will have as many lowercase
letters (a...z) as possible, but a solution will still satisfy all conditions.

\lstinputlisting{\CURPATH/1.py}

( The source code: \url{.../1.py} )

Let's try it on a small 350-bytes file\footnote{\url{.../cipher1.txt}}:

\begin{lstlisting}
% python 1.py cipher1.txt
len= 1
len= 2
len= 3
len= 4
len= 5

...

len= 16
len= 17
key=
00000000: 90 A0 21 52 48 84 FB FF  86 83 CF 50 46 12 7A F9  ..!RH......PF.z.
00000010: 36                                                6
plain=
00000000: 4D 72 2E 22 54 63 65 72  6F 6F 63 6D 27 48 6F 6C  Mr."Tceroocm'Hol
00000010: 6A 65 73 2C 22 70 63 6F  20 74 61 73 26 72 73 75  jes,"pco tas&rsu
00000020: 61 6B 6C 79 20 74 62 79  79 20 6F 61 74 63 27 69  akly tbyy oatc'i
00000030: 6E 20 73 68 65 20 6F 68  79 6E 69 6D 67 73 2A 27  n she ohynimgs*'
00000040: 73 61 76 62 0D 0A 75 72  68 65 20 74 6B 6F 73 63  savb..urhe tkosc
00000050: 27 6E 6F 74 27 69 6E 66  70 62 7A 75 65 6D 74 20  'not'infpbzuemt
00000060: 69 64 63 61 73 6E 6F 6E  73 22 70 63 65 6E 23 68  idcasnons"pcen#h
00000070: 65 26 70 61 73 20 72 70  20 61 6E 6B 2B 6E 69 64  e&pas rp ank+nid
00000080: 68 74 2A 27 77 61 73 27  73 65 61 76 62 6F 0D 0A  ht*'was'seavbo..
00000090: 62 74 20 72 6F 65 20 62  75 65 61 6B 64 66 78 74  bt roe bueakdfxt
000000A0: 20 77 61 62 6A 62 2E 20  49 27 73 74 6F 6D 63 2B   wabjb. I'stomc+
000000B0: 75 70 6C 6E 20 72 6F 65  20 68 62 61 72 74 6A 2A  upln roe hbartj*
000000C0: 79 75 67 23 61 6E 62 27  70 69 63 6C 65 64 20 77  yug#anb'picled w
000000D0: 77 2B 74 68 66 0D 0A 75  73 69 63 6B 27 77 68 69  w+thf..usick'whi
000000E0: 61 6F 2B 6F 75 71 20 76  6F 74 69 74 6F 75 20 68  ao+ouq votitou h
000000F0: 61 66 27 67 65 66 77 20  62 63 6F 69 6E 64 27 68  af'gefw bcoind'h
00000100: 69 6D 22 73 63 65 20 6D  69 67 6E 73 20 62 65 61  im"sce migns bea
00000110: 6F 72 65 2C 27 42 74 20  74 61 73 26 66 0D 0A 66  ore,'Bt tas&f..f
00000120: 6E 6E 65 2C 22 73 63 69  63 68 20 70 6F 62 63 65  nne,"scich pobce
00000130: 20 68 66 20 77 6D 68 6F  2C 20 61 75 6C 64 68 75   hf wmho, auldhu
00000140: 73 2D 6F 65 61 64 67 63  27 20 6F 65 20 74 6E 62  s-oeadgc' oe tnb
00000150: 20 73 6F 75 74 20 77 6A  6E 68 68 20 6A 73         sout wjnhh js
len= 18
len= 19
\end{lstlisting}

This is not readable. But what is interesting, the solution exist only for 17-byte key.

What do we know about English language texts?
Digits are rare there, so we can \textit{minimize} them in plain text.

There are so called \textit{digraphs}---a very popular combinations of two letters.
The most popular in English are: \textit{th}, \textit{he}, \textit{in} and \textit{er}.
We can count them in plain text and \textit{maximize} them:

\begin{lstlisting}
...

    # ... for each byte of plain text: 1 if the byte is digit:
    digits_in_plain=[Int('digits_in_plain_%d' % i) for i in range (cipher_len)]
    # ... for each byte of plain text: 1 if the byte + next byte is "th" characters:
    th_in_plain=[Int('th_in_plain_%d' % i) for i in range (cipher_len-1)]
    # ... etc:
    he_in_plain=[Int('he_in_plain_%d' % i) for i in range (cipher_len-1)]
    in_in_plain=[Int('in_in_plain_%d' % i) for i in range (cipher_len-1)]
    er_in_plain=[Int('er_in_plain_%d' % i) for i in range (cipher_len-1)]

...

    for i in range(cipher_len-1):
        # ... for each byte of plain text: 1 if the byte + next byte is "th" characters:
        s.add(th_in_plain[i]==(If(And(plain[i]==ord('t'),plain[i+1]==ord('h')), 1, 0)))
        # ... etc:
        s.add(he_in_plain[i]==(If(And(plain[i]==ord('h'),plain[i+1]==ord('e')), 1, 0)))
        s.add(in_in_plain[i]==(If(And(plain[i]==ord('i'),plain[i+1]==ord('n')), 1, 0)))
        s.add(er_in_plain[i]==(If(And(plain[i]==ord('e'),plain[i+1]==ord('r')), 1, 0)))

    # find solution, where the sum of all az_in_plain variables is maximum:
    s.maximize(Sum(*az_in_plain))
    # ... and sum of digits_in_plain is minimum:
    s.minimize(Sum(*digits_in_plain))

    # "maximize" presence of "th", "he", "in" and "er" digraphs:
    s.maximize(Sum(*th_in_plain))
    s.maximize(Sum(*he_in_plain))
    s.maximize(Sum(*in_in_plain))
    s.maximize(Sum(*er_in_plain))

...
\end{lstlisting}

( The source code: \url{.../2.py} )

Now this is something familiar:

\begin{lstlisting}
len= 17
key=
00000000: 90 A0 22 50 4F 8F FB FF  85 83 CF 56 41 12 7A FE  .."PO......VA.z.
00000010: 31                                                1
plain=
00000000: 4D 72 2D 20 53 68 65 72  6C 6F 63 6B 20 48 6F 6B  Mr- Sherlock Hok
00000010: 6D 65 73 2F 20 77 68 6F  20 77 61 73 20 75 73 75  mes/ who was usu
00000020: 66 6C 6C 79 23 76 65 72  79 20 6C 61 74 65 20 69  flly#very late i
00000030: 6E 27 74 68 65 23 6D 6F  72 6E 69 6E 67 73 2C 20  n'the#mornings,
00000040: 73 61 71 65 0D 0A 76 70  6F 6E 20 74 68 6F 73 65  saqe..vpon those
00000050: 20 6E 6F 73 20 69 6E 65  72 65 71 75 65 6E 74 20   nos inerequent
00000060: 6F 63 63 61 74 69 6F 6E  70 20 77 68 65 6E 20 68  occationp when h
00000070: 65 20 77 61 73 27 75 70  20 62 6C 6C 20 6E 69 67  e was'up bll nig
00000080: 68 74 2C 20 77 61 74 20  73 65 62 74 65 64 0D 0A  ht, wat sebted..
00000090: 61 74 20 74 68 65 20 65  72 65 61 68 66 61 73 74  at the ereahfast
000000A0: 20 74 61 62 6C 65 2E 20  4E 20 73 74 6C 6F 64 20   table. N stlod
000000B0: 75 70 6F 6E 20 74 68 65  20 6F 65 61 72 77 68 2D  upon the oearwh-
000000C0: 72 75 67 20 61 6E 64 20  70 69 64 6B 65 64 23 75  rug and pidked#u
000000D0: 70 20 74 68 65 0D 0A 73  74 69 63 6C 20 77 68 6A  p the..sticl whj
000000E0: 63 68 20 6F 75 72 20 76  69 73 69 74 68 72 20 68  ch our visithr h
000000F0: 62 64 20 6C 65 66 74 20  62 65 68 69 6E 63 20 68  bd left behinc h
00000100: 69 6E 20 74 68 65 20 6E  69 67 68 74 20 62 62 66  in the night bbf
00000110: 6F 72 66 2E 20 49 74 20  77 61 73 20 61 0D 0A 61  orf. It was a..a
00000120: 69 6E 65 2F 20 74 68 69  63 6B 20 70 69 65 63 65  ine/ thick piece
00000130: 27 6F 66 20 74 6F 6F 64  2C 20 62 75 6C 62 6F 75  'of tood, bulbou
00000140: 73 2A 68 65 61 67 65 64  2C 20 6F 66 20 74 68 65  s*heaged, of the
00000150: 20 73 68 72 74 20 74 68  69 63 68 20 69 73         shrt thich is
\end{lstlisting}

Several characters are wrong.
But we can fix them, adding these conditions:

\begin{lstlisting}
...
    # 3 known characters of plain text:
    s.add(plain[0xf]==ord('l'))
    s.add(plain[0x20]==ord('a'))
    s.add(plain[0x57]==ord('f'))
...
\end{lstlisting}

( The source code: \url{.../3.py} )

This key is seems correct:

\begin{lstlisting}
len= 17
key=
00000000: 90 A0 21 50 4F 8F FB FF  85 83 CF 56 41 12 7A F9  ..!PO......VA.z.
00000010: 31                                                1
plain=
00000000: 4D 72 2E 20 53 68 65 72  6C 6F 63 6B 20 48 6F 6C  Mr. Sherlock Hol
00000010: 6D 65 73 2C 20 77 68 6F  20 77 61 73 20 75 73 75  mes, who was usu
00000020: 61 6C 6C 79 20 76 65 72  79 20 6C 61 74 65 20 69  ally very late i
00000030: 6E 20 74 68 65 20 6D 6F  72 6E 69 6E 67 73 2C 20  n the mornings,
00000040: 73 61 76 65 0D 0A 75 70  6F 6E 20 74 68 6F 73 65  save..upon those
00000050: 20 6E 6F 74 20 69 6E 66  72 65 71 75 65 6E 74 20   not infrequent
00000060: 6F 63 63 61 73 69 6F 6E  73 20 77 68 65 6E 20 68  occasions when h
00000070: 65 20 77 61 73 20 75 70  20 61 6C 6C 20 6E 69 67  e was up all nig
00000080: 68 74 2C 20 77 61 73 20  73 65 61 74 65 64 0D 0A  ht, was seated..
00000090: 61 74 20 74 68 65 20 62  72 65 61 6B 66 61 73 74  at the breakfast
000000A0: 20 74 61 62 6C 65 2E 20  49 20 73 74 6F 6F 64 20   table. I stood
000000B0: 75 70 6F 6E 20 74 68 65  20 68 65 61 72 74 68 2D  upon the hearth-
000000C0: 72 75 67 20 61 6E 64 20  70 69 63 6B 65 64 20 75  rug and picked u
000000D0: 70 20 74 68 65 0D 0A 73  74 69 63 6B 20 77 68 69  p the..stick whi
000000E0: 63 68 20 6F 75 72 20 76  69 73 69 74 6F 72 20 68  ch our visitor h
000000F0: 61 64 20 6C 65 66 74 20  62 65 68 69 6E 64 20 68  ad left behind h
00000100: 69 6D 20 74 68 65 20 6E  69 67 68 74 20 62 65 66  im the night bef
00000110: 6F 72 65 2E 20 49 74 20  77 61 73 20 61 0D 0A 66  ore. It was a..f
00000120: 69 6E 65 2C 20 74 68 69  63 6B 20 70 69 65 63 65  ine, thick piece
00000130: 20 6F 66 20 77 6F 6F 64  2C 20 62 75 6C 62 6F 75   of wood, bulbou
00000140: 73 2D 68 65 61 64 65 64  2C 20 6F 66 20 74 68 65  s-headed, of the
00000150: 20 73 6F 72 74 20 77 68  69 63 68 20 69 73         sort which is
\end{lstlisting}

So this is correct 17-byte XOR-key.

Needless to say, that the bigger ciphertext for analysis we have, the better.
That 350-byte file is in fact the beginning of bigger file I prepared
(\url{https://github.com/DennisYurichev/SAT_SMT_article/blob/master/SMT/XOR/cipher2.txt}, 12903 bytes).
And a correct key for it can be found for it without additional \textit{heuristics} we used here.

SMT solver is overkill for this. I once solved this problem naively, and it was much faster:
\url{https://yurichev.com/blog/XOR_mask_2/}.
Nevertheless, this is yet another demonstration of yet another optimization problem.

The files: \url{.../XOR}.

