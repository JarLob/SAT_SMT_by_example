\subsection{Yet another explanation of NP-problems}

Various algorithms works "slow" or "fast" depending on the size of the input.

\subsubsection{Constant time, O(1)}

Time isn't dependent on size of input.
This is like \verb|string.size()| in C++ STL, given the fact that the implementation stores current string's size somewhere.
Good: \verb|.size()| method is fast, O(1), but during any modification of the string, a method(s) must update "size" field.

\subsubsection{Linear time, O(n)}

Time is dependent on size of input, linearly.
(Linear functions are functions which looks as lines on graph plot.)
This is search algorithms on linked lists, strings.
\verb|strlen()| is C.

\subsubsection{Hash tables and binary trees}

These are used in associative arrays implementations.
Using binary trees (\verb|std::map|, \verb|std::set| in C++ STL) is somewhat wasteful: you need too much memory, for each tree node.
Also, search speed is a binary logarithm of the size of tree: $O(log(n))$ (because depth of binary tree is $log_2(size\_of\_tree))$.
Good: keys are stored in sorted order, and you can retrieve them sorted while enumeration, by depth-first search of the binary tree.

Hash tables (\verb|std::unordered_map| and \verb|std::unordered_set| in C++ STL), on contrary, have constant access/insertion/deletion time
($O(1)$), and uses much less memory.
This is because all key/value pairs are stored in a (big) table, and a hashed key is an index inside the table.
Hash function should be good to make keys which are close to each other (100,101,102...) distributed uniformly across the table.
However, you can't retrieve all keys from hash table in sorted form.
It's a good idea to use hash tables instead of binary trees, if you don't need keys sorted.

\subsubsection{EXPTIME}

Time is dependent exponentially on size of input, $O(2^{p(n)})$ or just $O(2^n)$.
This is bruteforce.
When you try to break some cryptoalgorithm by bruteforcing and trying all possible inputs.
For 8 bits of input, there are $2^8=256$ possible values, etc.
Bruteforce can crack any cryptoalgorithm and solve almost any problem, but this is not very interesting, because it's extremely slow.

\subsubsection{NP-problems}

Finding correct inputs for CNF formula is an NP-problem, but you can also find them using simple bruteforce.
In fact, SAT-solvers also do bruteforce, but the resulting search tree is extremely big.
And to make search faster, SAT-solvers prune search tree as early as possible, but in unpredictable ways.
This makes execution time of SAT-solvers also unpredictable.
In fact, this is a problem, to predict how much time it will take to find a solution for CNF formula.
No SAT/SMT solvers can say you \ac{ETA} of their work.

NP problems has no O-notation, meaning, there is no link between size (and contents) of input and execution time.

In contrast to other problems listed here (constant, linear, logarithmical, exponential), you can predict time of work by observing
input values.
You can predict, how long it will take to sort items by looking at a list of items to be sorted.
This is a problem for NP-problems: you can't predict it.

Probably, you could devise faster algorithm to solve NP-problems, maybe working 1000 times faster, but still, it will work unpredictably.

Also, both SAT and SMT solvers are very capricious to input data.
This is a rare case, when shotgun debugging
\footnote{\url{http://www.catb.org/jargon/html/S/shotgun-debugging.html} \url{https://en.wikipedia.org/wiki/Shotgun_debugging}}
is justified!

\subsubsection{P=NP?}

One of the famous problem asking, if it's possible to solve NP-problems fast.

In other words, a scientist (or anyone else) who will find a way to solve NP-problems in predictable time (but faster than bruteforce/EXPTIME),
will make a significant or revolutionary step in computer science.

In popular opinion, this will crack all cryptoalgorithms we currently use.
In fact, this is not true.
Some scientists, like Donald Knuth, believe, that there may be a way to solve NP-problems in polynomial time, but still too
slow to be practical.

However, majority of scientists believe the humankind is not ready for such problems.

\subsubsection{What are SAT/SMT solvers}

To my taste, these are like APIs to NP-problems.
Maybe not the best, but good enough.
(Other APIs are, at least, CSPLIB, Prolog programming language...)

Because all NP-problems can be reduced (or converted) to SAT problems, or represented as SAT problems.

This is what I do in this book: I'm solving many well-known NP-problems using SAT/SMT solvers.

