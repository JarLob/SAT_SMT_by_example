\subsection{Package manager and Z3}

Here is simplified example.
We have libA, libB, libC and libD, available in various versions (and flavors).
We're going to install programA and programB, which use these libraries.

\lstinputlisting[style=custompy]{other/dep/dependency.py}

( The source code: \url{https://github.com/DennisYurichev/SAT_SMT_by_example/blob/master/other/dep/dependency.py} )

The output:

\begin{lstlisting}
sat
[libB = 5,
 libD = 999,
 libC = 10,
 programB = 7,
 programA = 1,
 libA = 2]
\end{lstlisting}

999 means that there is no need to install libD, it's not required by other packages.

Change version of ProgramB to v8 and it will says ``unsat'', meaning, there is a conflict:
ProgramA requires libA v2, but ProgramB v8 eventually requires newer libA.

Still, there is a work to do: ``unsat'' message is somewhat useless to end user,
some information about conflicting items should be printed.

Here is my another optimization problem example: \ref{set_cover}.

More about using SAT/SMT solvers in package managers: \url{https://research.swtch.com/version-sat},
\url{https://cseweb.ucsd.edu/~lerner/papers/opium.pdf}.

Now in the opposite direction: forcing aptitude package manager to solve Sudoku: \\
\url{http://web.archive.org/web/20160326062818/http://algebraicthunk.net/~dburrows/blog/entry/package-management-sudoku/}.

Some readers may ask, how to order libraries/programs/packages to be installed?
This is simpler problem, which is often solved by topological sorting.
The algorithm reorders graph in such a way so that vertices not depended on anything will be on the top of queue.
Next, there will be vertices dependend on vertices from the previous layer.
And so on.

\textit{make} UNIX utility does this while constructing order of items to be processed.
Even more: older \textit{make} utilities offloaded the job to the external utility (\textit{tsort}).
Some older UNIX has it, at least some versions of NetBSD
\footnote{\url{http://netbsd.gw.com/cgi-bin/man-cgi/man?tsort+1+NetBSD-current}}.

