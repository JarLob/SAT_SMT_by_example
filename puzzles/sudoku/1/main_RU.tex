\subsubsection{Простая Судоку и SMT}
\label{sudoku_SMT}

\paragraph{Первая идея}

Всё что нужно решить, это как определять в одном выражении, содержат ли 9 переменных все 9 уникальных чисел?
Они ведь не упорядочены и не отсортированы, все-таки.

Из школьной арифметики, мы можем найти такую идею:

\begin{equation}
\underbrace{10^{i_1} + 10^{i_2} + \cdots + 10^{i_9}}_9 = 1111111110
\end{equation}

Берете каждую входную переменную, вычисляете $10^i$ и суммируете.
Если все входные значения уникальны, каждая найдет свое собственное место.
И даже более того: не будет дыр, т.е., не будет пропущенных значений.
Так что, в случае Судоку, число 1111111110 будет конечным результатом, означающим, что все входные
9 переменных уникальны, в пределах 1..9.

Возведение в степень это тяжелая операция, можно ли использовать двоичные операции? Да, просто замените 10 на 2:

\begin{equation}
\underbrace{2^{i_1} + 2^{i_2} + \cdots + 2^{i_9}}_9 = 1111111110_2
\end{equation}

Эффект тот же, но результат будет в двоичной системе а не в десятичной.

Вот рабочий пример:

\lstinputlisting{puzzles/sudoku/1/sudoku_plus_Z3.py}
( \url{https://github.com/DennisYurichev/SAT_SMT_article/.../sudoku_plus_Z3.py} )

\begin{lstlisting}
% time python sudoku_plus_Z3.py
1 4 5 3 2 7 6 9 8
8 3 9 6 5 4 1 2 7
6 7 2 9 1 8 5 4 3
4 9 6 1 8 5 3 7 2
2 1 8 4 7 3 9 5 6
7 5 3 2 9 6 4 8 1
3 6 7 5 4 2 8 1 9
9 8 4 7 6 1 2 3 5
5 2 1 8 3 9 7 6 4

real    0m11.717s
user    0m10.896s
sys     0m0.068s
\end{lstlisting}

И даже более того, можно заменить суммирование на логическое ИЛИ:

\begin{equation}
\underbrace{2^{i_1} \vee 2^{i_2} \vee \cdots \vee 2^{i_9}}_9 = 1111111110_2
\end{equation}

% FIXME: только часть исходника
\lstinputlisting{puzzles/sudoku/1/sudoku_or_Z3.py}
( \url{https://github.com/DennisYurichev/SAT_SMT_article/.../sudoku_or_Z3.py} )

Теперь работает намного быстрее. Наверное, Z3 лучше поддерживает операцию ИЛИ над битовыми векторами, чем сложение?

\begin{lstlisting}
% time python sudoku_or_Z3.py
1 4 5 3 2 7 6 9 8
8 3 9 6 5 4 1 2 7
6 7 2 9 1 8 5 4 3
4 9 6 1 8 5 3 7 2
2 1 8 4 7 3 9 5 6
7 5 3 2 9 6 4 8 1
3 6 7 5 4 2 8 1 9
9 8 4 7 6 1 2 3 5
5 2 1 8 3 9 7 6 4

real    0m1.429s
user    0m1.393s
sys     0m0.036s
\end{lstlisting}

Головоломка, которую я использовал как пример, известна как самая трудная из известных
\footnote{\url{http://www.mirror.co.uk/news/weird-news/worlds-hardest-sudoku-can-you-242294}} (по крайней мере для людей).
Для решения понадобилось $\approx 1.4$ секунды на моем ноутбуке Intel Core i3-3110M 2.4GHz.

\paragraph{Вторая идея}

Мой первый подход далек от эффективного, я сделал то что первым пришло в голову, и оно заработало.
Другой подход это использовать команду \TT{distinct} в SMT-LIB, которая говорит Z3, что некоторые переменные
должны быть отличны друг от друга (или уникальны).
Эта команда также имеется в питоновском интерфейсе к Z3.

Я переписал мой первый солвер Судоку, теперь он работает над \textit{sort}-ом 
\textit{Int}, и имеет команды \TT{distinct} вместо битовых операций,
и еще один констрайнт добавлен: значение каждой ячейки должно быть в пределах 1..9, потому что, иначе, Z3 предложит
(хотя и корректное) решение с очень большими и/или отрицательными числами.

% FIXME: только часть исходника
\lstinputlisting{puzzles/sudoku/1/sudoku2_Z3.py}
( \url{https://github.com/DennisYurichev/SAT_SMT_article/.../sudoku2_Z3.py} )

\begin{lstlisting}
% time python sudoku2_Z3.py
1 4 5 3 2 7 6 9 8
8 3 9 6 5 4 1 2 7
6 7 2 9 1 8 5 4 3
4 9 6 1 8 5 3 7 2
2 1 8 4 7 3 9 5 6
7 5 3 2 9 6 4 8 1
3 6 7 5 4 2 8 1 9
9 8 4 7 6 1 2 3 5
5 2 1 8 3 9 7 6 4

real    0m0.382s
user    0m0.346s
sys     0m0.036s
\end{lstlisting}

Это намного быстрее.

\paragraph{Вывод}

\ac{SMT}-солверы настолько удобны, что в нашем солвере Судоку нет ничего больше ничего, мы просто определили
отношения между переменными (ячейками).

\paragraph{Домашная работа}

Как видно, настоящая головоломка Судоку это та, для которой имеется только одно решение.
Фрагмент кода, который приведен здесь, показывает только первое.
Использая метод описанный раннее (\ref{SMTEnumerate}, также называемый ``подсчет моделей (model counting)''), 
попытайтесь найти больше решений, или доказать, что решение, которое вы нашли, единственное возможное.

\paragraph{Дальнейшее чтение}

\url{http://www.norvig.com/sudoku.html}

\paragraph{Судоку как \ac{SAT}-проблема}

Головоломку Судоку можно также представить как огромное \ac{CNF}-уравнение и использовать \ac{SAT}-солвер для поиска решения,
но это просто сложнее.

Некоторые статьи об этом:
\textit{Building a Sudoku Solver with SAT}\footnote{\url{https://dspace.mit.edu/bitstream/handle/1721.1/106923/6-005-fall-2011/contents/assignments/MIT6_005F11_ps4.pdf}},
Tjark Weber, \textit{A SAT-based Sudoku Solver}\footnote{\url{https://www.lri.fr/~conchon/mpri/weber.pdf}},
Ines Lynce, Joel Ouaknine, \textit{Sudoku as a SAT Problem}\footnote{\url{http://sat.inesc-id.pt/~ines/publications/aimath06.pdf}},
Gihwon Kwon, Himanshu Jain, \textit{Optimized CNF Encoding for Sudoku Puzzles}\footnote{\url{http://www.cs.cmu.edu/~hjain/papers/sudoku-as-SAT.pdf}}.

\ac{SMT}-солвер также может использовать \ac{SAT}-солвер в своем ядре, так что он делает всю эту скучную работу
по трансляции.
Хотя и, как и компилятор, он может это делать не самым эффективным способом.

