\subsection{KLEE}

I've also rewritten Sudoku example (\ref{sudoku_SMT}) for KLEE:

\lstinputlisting[numbers=left]{puzzles/sudoku/KLEE/klee_sudoku_or1.c}

Let's run it:

\begin{lstlisting}
% clang -emit-llvm -c -g klee_sudoku_or1.c
...

\$ time klee klee_sudoku_or1.bc
KLEE: output directory is "/home/klee/klee-out-98"
KLEE: WARNING: undefined reference to function: klee_assert
KLEE: WARNING ONCE: calling external: klee_assert(0)
KLEE: ERROR: /home/klee/klee_sudoku_or1.c:93: failed external call: klee_assert
KLEE: NOTE: now ignoring this error at this location

KLEE: done: total instructions = 7512
KLEE: done: completed paths = 161
KLEE: done: generated tests = 161

real    3m44.111s
user    3m43.319s
sys     0m0.951s
\end{lstlisting}

Now this is really slower (on my Intel Core i3-3110M 2.4GHz notebook) in comparison to Z3Py solution (\ref{sudoku_SMT}).

But the answer is correct:

\begin{lstlisting}
% ls klee-last | grep err
test000161.external.err

% ktest-tool --write-ints klee-last/test000161.ktest
ktest file : 'klee-last/test000161.ktest'
args       : ['klee_sudoku_or1.bc']
num objects: 1
object    0: name: b'cells'
object    0: size: 81
object    0: data: b'\x01\x04\x05\x03\x02\x07\x06\t\x08\x08\x03\t\x06\x05\x04\x01\x02\x07\x06\x07\x02\t\x01\x08\x05\x04\x03\x04\t\x06\x01\x08\x05\x03\x07\x02\x02\x01\x08\x04\x07\x03\t\x05\x06\x07\x05\x03\x02\t\x06\x04\x08\x01\x03\x06\x07\x05\x04\x02\x08\x01\t\t\x08\x04\x07\x06\x01\x02\x03\x05\x05\x02\x01\x08\x03\t\x07\x06\x04'
\end{lstlisting}

Character \TT{\textbackslash{}t} has code of 9 in C/C++,
and KLEE prints byte array as a C/C++ string, so it shows some values in such way.
We can just keep in mind that there is 9 at the each place where we see \TT{\textbackslash{}t}.
The solution, while not properly formatted, correct indeed.

By the way, at lines 42 and 43 you may see how we tell to KLEE that all array elements must be within some limits.
If we comment these lines out, we've got this:

\begin{lstlisting}
% time klee klee_sudoku_or1.bc
KLEE: output directory is "/home/klee/klee-out-100"
KLEE: WARNING: undefined reference to function: klee_assert
KLEE: ERROR: /home/klee/klee_sudoku_or1.c:51: overshift error
KLEE: NOTE: now ignoring this error at this location
KLEE: ERROR: /home/klee/klee_sudoku_or1.c:51: overshift error
KLEE: NOTE: now ignoring this error at this location
KLEE: ERROR: /home/klee/klee_sudoku_or1.c:51: overshift error
KLEE: NOTE: now ignoring this error at this location
...
\end{lstlisting}

KLEE warns us that shift value at line 51 is too big.
Indeed, KLEE may try all byte values up to 255 (0xFF), which are pointless to use there,
and may be a symptom of error or bug, so KLEE warns about it. % FIXME chk spelling

Now let's use \TT{klee\_assume()} again:

\lstinputlisting{puzzles/sudoku/KLEE/klee_sudoku_or2.c}

\begin{lstlisting}
% time klee klee_sudoku_or2.bc
KLEE: output directory is "/home/klee/klee-out-99"
KLEE: WARNING: undefined reference to function: klee_assert
KLEE: WARNING ONCE: calling external: klee_assert(0)
KLEE: ERROR: /home/klee/klee_sudoku_or2.c:93: failed external call: klee_assert
KLEE: NOTE: now ignoring this error at this location

KLEE: done: total instructions = 7119
KLEE: done: completed paths = 1
KLEE: done: generated tests = 1

real    0m35.312s
user    0m34.945s
sys     0m0.318s
\end{lstlisting}

That works much faster: perhaps KLEE indeed handle this \textit{intrinsic} in a special way.
And, as we see, the only one path has been found (one we actually interesting in it) instead of 161.

It's still much slower than Z3Py solution, though.

