\section{Art of problem solving}

\url{http://artofproblemsolving.com/wiki/index.php?title=Mock_AIME_2_2006-2007_Problems/Problem_8}:

\begin{framed}
\begin{quotation}
The positive integers $x_1, x_2, ... , x_7$ satisfy $x_6 = 144$ and $x_{n+3} = x_{n+2}(x_{n+1}+x_n)$ for $n = 1, 2, 3, 4$. Find the last three digits of $x_7$.
\end{quotation}
\end{framed}

This is it:

\begin{lstlisting}[style=custompy]
from z3 import *

s=Solver()

x1, x2, x3, x4, x5, x6, x7=Ints('x1 x2 x3 x4 x5 x6 x7')

s.add(x1>=0)
s.add(x2>=0)
s.add(x3>=0)
s.add(x4>=0)
s.add(x5>=0)
s.add(x6>=0)
s.add(x7>=0)

s.add(x6==144)

s.add(x4==x3*(x2+x1))
s.add(x5==x4*(x3+x2))
s.add(x6==x5*(x4+x3))
s.add(x7==x6*(x5+x4))

# get all results:

results=[]
while True:
    if s.check() == sat:
        m = s.model()
        print m

        results.append(m)
        block = []
        for d in m:
            c=d()
            block.append(c != m[d])
        s.add(Or(block))
    else:
        print "total results", len(results)
        break
\end{lstlisting}

Two solutions possible, but in both x7 is ending by 456:

\begin{lstlisting}
[x2 = 1,
 x3 = 1,
 x1 = 7,
 x4 = 8,
 x5 = 16,
 x7 = 3456,
 x6 = 144]
[x3 = 2,
 x2 = 1,
 x1 = 2,
 x6 = 144,
 x4 = 6,
 x5 = 18,
 x7 = 3456]
total results 2
\end{lstlisting}

