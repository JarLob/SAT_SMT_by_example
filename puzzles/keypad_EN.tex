\subsection{Recreational math, calculator's keypad and divisibility}

I've once read about this puzzle.
Imagine calculator's keypad:

\begin{lstlisting}
789
456
123
\end{lstlisting}

If you form any rectangle or square out of keys, like:

\begin{lstlisting}
 7 8 9
+---+
|4 5|6
|1 2|3
+---+
\end{lstlisting}

The number is 4521. Or 2145, or 5214.
All these numbers are divisible by 11, 111 and 111.
One explanation: \url{https://files.eric.ed.gov/fulltext/EJ891796.pdf}.

However, I could try to prove that all these numbers are indeed divisible.

\begin{lstlisting}[style=custompy]

from z3 import *

"""
We will keep track on numbers using row/col representation:

 |0 1 2 <-col
-|- - -
0|7 8 9
1|4 5 6
2|1 2 3
^
|
row

"""

# map coordinates to number on keypad:
def coords_to_num (r, c):
    return If(And(r==0, c==0), 7,
    If(And(r==0, c==1), 8,
    If(And(r==0, c==2), 9,
    If(And(r==1, c==0), 4,
    If(And(r==1, c==1), 5,
    If(And(r==1, c==2), 6,
    If(And(r==2, c==0), 1,
    If(And(r==2, c==1), 2,
    If(And(r==2, c==2), 3, 9999)))))))))

s=Solver()

# coordinates of upper left corner:
from_r, from_c = Ints('from_r from_c')
# coordinates of bottom right corner:
to_r, to_c = Ints('to_r to_c')

# all coordinates are in [0..2]:
s.add(And(from_r>=0, from_r<=2, from_c>=0, from_c<=2))
s.add(And(to_r>=0, to_r<=2, to_c>=0, to_c<=2))

# bottom-right corner is always under left-upper corner, or equal to it, or to the right of it:
s.add(to_r>=from_r)
s.add(to_c>=from_c)

# numbers on keypads for all 4 corners:
LT, RT, BL, BR = Ints('LT RT BL BR')

# ... which are:
s.add(LT==coords_to_num(from_r, from_c))
s.add(RT==coords_to_num(from_r, to_c))
s.add(BL==coords_to_num(to_r, from_c))
s.add(BR==coords_to_num(to_r, to_c))

# 4 possible 4-digit numbers formed by passing by 4 corners:
n1, n2, n3, n4 = Ints('n1 n2 n3 n4')

s.add(n1==LT*1000 + RT*100 + BR*10 + BL)
s.add(n2==RT*1000 + BR*100 + BL*10 + LT)
s.add(n3==BR*1000 + BL*100 + LT*10 + RT)
s.add(n4==BL*1000 + LT*100 + RT*10 + BR)

# what we're going to do?
prove=False
enumerate_rectangles=True

assert prove != enumerate_rectangles

if prove:
    # prove by finding counterexample.
    # find any variable state for which remainder will be non-zero:
    s.add(And((n1%11) != 0), (n1%111) != 0, (n1%1111) != 0)
    s.add(And((n2%11) != 0), (n2%111) != 0, (n2%1111) != 0)
    s.add(And((n3%11) != 0), (n3%111) != 0, (n3%1111) != 0)
    s.add(And((n4%11) != 0), (n4%111) != 0, (n4%1111) != 0)

    # this is impossible, we're getting unsat here, because no counterexample exist:
    print s.check()

# ... or ...

if enumerate_rectangles:
    # enumerate all possible solutions:
    results=[]
    while True:
        if s.check() == sat:
            m = s.model()
            #print_model(m)
            print m
            print m[n1]
            print m[n2]
            print m[n3]
            print m[n4]
            results.append(m)
            block = []
            for d in m:
                c=d()
                block.append(c != m[d])
            s.add(Or(block))
        else:
            print "results total=", len(results)
            break

\end{lstlisting}

Enumeration. only 36 rectangles exist on 3*3 keypad:

\begin{lstlisting}
[n1 = 7821,
 BL = 1,
 n2 = 8217,
 to_r = 2,
 LT = 7,
 RT = 8,
 BR = 2,
 n4 = 1782,
 from_r = 0,
 n3 = 2178,
 from_c = 0,
 to_c = 1]
7821
8217
2178
1782
[n1 = 7931,
 BL = 1,
 n2 = 9317,
 to_r = 2,
 LT = 7,
 RT = 9,
 BR = 3,
 n4 = 1793,
 from_r = 0,
 n3 = 3179,
 from_c = 0,
 to_c = 2]
7931
9317
3179
1793

...

[n1 = 5522,
 BL = 2,
 n2 = 5225,
 to_r = 2,
 LT = 5,
 RT = 5,
 BR = 2,
 n4 = 2552,
 from_r = 1,
 n3 = 2255,
 from_c = 1,
 to_c = 1]
5522
5225
2255
2552
results total= 36
\end{lstlisting}

