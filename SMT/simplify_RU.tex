\subsection{Сокращение длинных и запутанных выражений используя Mathematica и Z3}

\dots которые могут быть результатом Hex-Rays и/или ручного переписывания.

В свою книгу RE4B я добавлял о возможностях Wolfram Mathematica минимизировать выражения
\footnote{\url{https://github.com/DennisYurichev/RE-for-beginners/blob/cd85356051937e87f90967cc272248084808223b/other/hexrays_EN.tex\#L412}, \url{https://beginners.re/}}.

Сегодня я наткнулся на такой вывод Hex-Rays:

\begin{lstlisting}
if ( ( x != 7 || y!=0 ) && (x < 6 || x > 7) )
{
        ...
};
\end{lstlisting}

И Mathematica и Z3 (используя команду ``simplify'') не могут сделать его проще, но у меня чувство, что есть тут что-то
избыточное.

Посмотрим на правую часть выражения.
Если $x$ должно быть меньше чем \textit{ИЛИ} больше чем 7, тогда она может содержать любую переменную кроме 6 \textit{И} 7, верно?
Так что я могу переписать это вручную:

\begin{lstlisting}
if ( ( x != 7 || y!=0 ) && x != 6 && x != 7) )
{
        ...
};
\end{lstlisting}

А это уже то, что Mathematica может упростить:

\begin{lstlisting}
In[]:= BooleanMinimize[(x != 7 || y != 0) && (x != 6 && x != 7)]
Out[]:= x != 6 && x != 7
\end{lstlisting}

$y$ сокращается.

Но действительно ли я прав?
И почему Mathematica и Z3 не упростили это в самом начале?

Я могу использовать Z3 чтобы доказать, что эти выражения равны друг другу:

\begin{lstlisting}
#!/usr/bin/env python

from z3 import *

x=Int('x')
y=Int('y')

s=Solver()

exp1=And(Or(x!=7, y!=0), Or(x<6, x>7))
exp2=And(x!=6, x!=7)

s.add(exp1!=exp2)

print simplify(exp1) # no luck

print s.check()
print s.model()
\end{lstlisting}

Z3 не может найти контрпример, так что он говорит ``unsat'', означая, что эти выражения эквивалентны друг другу.
Так что я переписал это выражение в своем коде, тесты прошли успешно, итд.

Да, использовать и Mathematica и Z3 это слишком, и это простая булева алгебра,
но после \textasciitilde{}10 часов за компьютером, вы можете делать по-настоящему тупые ошибки,
и никогда не помешает дополнительное доказательство, что фрагмент кода действительно корректен.

