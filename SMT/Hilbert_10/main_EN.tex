\subsection{Hilbert's 10th problem, Fermat’s last theorem and SMT solvers}

Hilbert's 10th problem states that you cannot devise an algorithm which can solve any diophantine equation over integers.
However, it's important to understand, that this is possible over fixed-size bitvectors.

Fermat's last theorem states that there are no integer solution(s) for $a^n + b^n = c^n$, for $n>=3$.

Let's prove it for n=3 and for a in 0..255 range:

\lstinputlisting[style=custompy]{SMT/Hilbert_10/fermat.py}

Z3 gives "unsat", meaning, it couldn't find any a/b/c.
However, this is possible to check even using brute-force search.

If to replace "BitVecs" by "Ints", Z3 would give "unknown":

\lstinputlisting[style=custompy]{SMT/Hilbert_10/fermat2.py}

In short: anything is decidable (you can build an algorithm which can solve equation or not) under fixed-size bitvectors.
Given enough computational power, you can solve such equations for big bit-vectors.
But this is not possible for integers or bit-vectors of any size.

Another interesting reading about this by Leonardo de Moura:
\url{https://stackoverflow.com/questions/13898175/how-does-z3-handle-non-linear-integer-arithmetic}.

