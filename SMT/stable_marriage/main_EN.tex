\subsection{Stable marriage problem}

See also in
\href{https://en.wikipedia.org/wiki/Stable_marriage_problem}{Wikipedia} and 
\href{https://rosettacode.org/wiki/Stable_marriage_problem}{Rosetta code}.

Layman's explanation in Russian: \url{https://lenta.ru/articles/2012/10/15/nobel/}.

My solution is much less efficient, because much simpler/better algorithm exists (Gale/Shapley algorithm),
but I did it to demonstrate the essence of the problem plus as a yet another SMT-solvers and Z3 demonstration.

See comments:

\lstinputlisting{SMT/stable_marriage/stable.py}

( The source code: \url{URL/stable.py} )

Result is seems to be correct:

\begin{lstlisting}
sat

ManChoice:
abe <-> ivy
bob <-> cath
col <-> dee
dan <-> fay
ed <-> jan
fred <-> bea
gav <-> gay
hal <-> eve
ian <-> hope
jon <-> abi

WomanChoice:
abi <-> jon
bea <-> fred
cath <-> bob
dee <-> col
eve <-> hal
fay <-> dan
gay <-> gav
hope <-> ian
ivy <-> abe
jan <-> ed
\end{lstlisting}

This is what we did in plain English language.
``Connect men and women somehow, we don't care how.
But no pair must exist of those who prefer each other (simultaneously) over their current spouses''.
Gale/Shapley algorithm uses ``steps'' to ``stabilize'' marriage.
There are no ``steps'', all pairs are married couples already.

Another important thing to notice: only one solution must exist.

\begin{lstlisting}
...

results=[]

# enumerate all possible solutions:
while True:
    if s.check() == sat:
        m = s.model()
        #print m
        results.append(m)
        block = []
        for d in m:
            c=d()
            block.append(c != m[d])
        s.add(Or(block))
    else:
        print "results total=", len(results)
        break

...
\end{lstlisting}

( The source code: \url{URL/stable2.py} )

That reports only 1 model available, which is correct indeed.

