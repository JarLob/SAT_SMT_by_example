\subsection{Дискретная томография}

Как компьютерная томография (КТ) вообще работает?
Тело человека бомбардируется рентгеновскими лучами под разными углами во вращающемся торе.
Детекторы излучения также находятся в торе и вся информация записывается.

Мы тут можем симулировать простой томограф.
Символ ``i'' вращается и будет ``просвечен'' под 4-я углами.
Представим что символ бомбардируется рентгеновскими лучами слева.
Все звездочки в каждом ряду суммируются и сумма ``принимается'' рентгеновскими детекторами справа.

\begin{lstlisting}
WIDTH= 11 HEIGHT= 11
angle=(π/4)*0
    **      2
    **      2
            0
   ***      3
    **      2
    **      2
    **      2
    **      2
    **      2
   ****     4
            0
[2, 2, 0, 3, 2, 2, 2, 2, 2, 4, 0] ,
angle=(π/4)*1
            0
            0
  *         1
 **         2
    *       1
    **      2
     **     2
     ****   4
       *    1
      *     1
            0
[0, 0, 1, 2, 1, 2, 2, 4, 1, 1, 0] ,
angle=(π/4)*2
            0
            0
            0
            0
         *  1
** *******  9
** *******  9
   *     *  2
            0
            0
            0
[0, 0, 0, 0, 1, 9, 9, 2, 0, 0, 0] ,
angle=(π/4)*3
            0
            0
       *    1
       **   2
      ** *  3
     ***    3
    **      2
            0
  **        2
   *        1
            0
[0, 0, 1, 2, 3, 3, 2, 0, 2, 1, 0] ,
\end{lstlisting}

( Исходный код: \url{https://github.com/DennisYurichev/SAT_SMT_article/blob/master/SMT/tomo/gen.py} )

Все что мы получаем из нашего игрушечного томографа это 4 вектора, это суммы всех звездочек в рядях для 4-х углов:

\begin{lstlisting}
[2, 2, 0, 3, 2, 2, 2, 2, 2, 4, 0] ,
[0, 0, 1, 2, 1, 2, 2, 4, 1, 1, 0] ,
[0, 0, 0, 0, 1, 9, 9, 2, 0, 0, 0] ,
[0, 0, 1, 2, 3, 3, 2, 0, 2, 1, 0] ,
\end{lstlisting}

Как восстановить изначальное изображение?
Мы собираемся представить матрицу 11*11, где сумма каждого ряда должна равняться некоторому известному нам значению.
Затем мы вращаем матрицу и делаем это снова.

Для первой матрицы, система уравнений выглядит так (мы добавляем сюда значения из первого вектора):

\begin{lstlisting}
C1,1 + C1,2 + C1,3 + ... + C1,11 =      2
C2,1 + C2,2 + C2,3 + ... + C2,11 =      2

...

C10,1 + C10,2 + C10,3 + ... + C10,11 =  4
C11,1 + C11,2 + C11,3 + ... + C11,11 =  0
\end{lstlisting}

Мы строим также подобны системы уравнений для каждого угла.

Ф-ция ``rotate'' была взята из программы для генерации, потому что, вследствии динамической типизации в Питоне,
не важно, чем ф-ция оперирует: строки, символы, или объекты в Z3 содержащие переменные, так что она одинаково хорошо
работает  для всех.

\begin{lstlisting}
#-*- coding: utf-8 -*-

import math, sys
from z3 import *

# https://en.wikipedia.org/wiki/Rotation_matrix
def rotate(pic, angle):
    WIDTH=len(pic[0])
    HEIGHT=len(pic)
    #print WIDTH, HEIGHT
    assert WIDTH==HEIGHT
    ofs=WIDTH/2

    out = [[0 for x in range(WIDTH)] for y in range(HEIGHT)]

    for x in range(-ofs,ofs):
        for y in range(-ofs,ofs):
            newX = int(round(math.cos(angle)*x - math.sin(angle)*y,3))+ofs
            newY = int(round(math.sin(angle)*x + math.cos(angle)*y,3))+ofs
            # clip at boundaries, hence min(..., HEIGHT-1)
            out[min(newX,HEIGHT-1)][min(newY,WIDTH-1)]=pic[x+ofs][y+ofs]
    return out

vectors=[
[2, 2, 0, 3, 2, 2, 2, 2, 2, 4, 0] ,
[0, 0, 1, 2, 1, 2, 2, 4, 1, 1, 0] ,
[0, 0, 0, 0, 1, 9, 9, 2, 0, 0, 0] ,
[0, 0, 1, 2, 3, 3, 2, 0, 2, 1, 0]]

WIDTH = HEIGHT = len(vectors[0])

s=Solver()
cells=[[Int('cell_r=%d_c=%d' % (r,c)) for c in range(WIDTH)] for r in range(HEIGHT)]

# monochrome picture, only 0's or 1's:
for c in range(WIDTH):
    for r in range(HEIGHT):
        s.add(Or(cells[r][c]==0, cells[r][c]==1))

def all_zeroes_in_vector(vec):
    for v in vec:
        if v!=0:
            return False
    return True

ANGLES=len(vectors)
for a in range(ANGLES):
    angle=a*(math.pi/ANGLES)
    rows=rotate(cells, angle)
    r=0
    for row in rows:
        # skip empty rows:
        if all_zeroes_in_vector(row)==False:
            # sum of row must be equal to the corresponding element of vector:
            s.add(Sum(*row)==vectors[a][r])
        r=r+1

print s.check()
m=s.model()
for r in range(HEIGHT):
    for c in range(WIDTH):
        if str(m[cells[r][c]])=="1":
            sys.stdout.write("*")
        else:
            sys.stdout.write(" ")
    print ""
\end{lstlisting}

( Исходный код: \url{https://github.com/DennisYurichev/SAT_SMT_article/blob/master/SMT/tomo/solve.py} )

Это работает:

\begin{lstlisting}
% python solve.py
sat
    **
    **

   ***
    **
    **
    **
    **
    **
   ****
\end{lstlisting}

Другими словами, все что делает SMT-солвер это решает систему уравнений.

Так что, 4-х углов достаточно.
Что если бы мы использовали только 3 угла?

\begin{lstlisting}
WIDTH= 11 HEIGHT= 11
angle=(π/3)*0
    **      2
    **      2
            0
   ***      3
    **      2
    **      2
    **      2
    **      2
    **      2
   ****     4
            0
[2, 2, 0, 3, 2, 2, 2, 2, 2, 4, 0] ,
angle=(π/3)*1
            0
            0
            0
 **         2
 **         2
   ***      3
     ****   4
       **   2
       *    1
            0
            0
[0, 0, 0, 2, 2, 3, 4, 2, 1, 0, 0] ,
angle=(π/3)*2
            0
            0
            0
       **   2
       **   2
     *****  5
    **      2
 **         2
  *         1
            0
            0
[0, 0, 0, 2, 2, 5, 2, 2, 1, 0, 0] ,
\end{lstlisting}

Нет, этого не достаточно:

\begin{lstlisting}
% time python solve3.py
sat
 *  *
    **

     * **
   **
   *  *
    **
     *   *
*   *
   ****
\end{lstlisting}

Впрочем, результат корректный, но 3 вектора допускают слишком много возможных ``исходных изображений'', и Z3 нашел
первое.

Дальнейшее чтение:
\url{https://en.wikipedia.org/wiki/Discrete_tomography},
\url{https://en.wikipedia.org/wiki/2-satisfiability#Discrete_tomography}.

