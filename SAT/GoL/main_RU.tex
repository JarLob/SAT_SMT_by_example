\subsection{Игра ``Жизнь'' Конвея}

\subsubsection{Откатывание состояния игры ``Жизнь'' назад}

Можно ли откатить назад известное состояние игры?
Это можно решить брутфорсом, но это экстремально медленно и неэффективно.

Попробуем SAT-солвер.

В начале, нужно определить ф-цию, которая будет говорить, будет ли новая клетка создаваться/рождаться, сохраняться/оставаться
или умирать.
Краткое напоминание: клетка рождается, если у нее 3 соседа, она остается живой, если у нее 2 или 3 соседа, она умирает
в остальных случаях.

Вот как я могу определить ф-цию, отражающую состояние новой клетки для следующего состояния:

\begin{lstlisting}
if center==true:
	return popcnt2(neighbours) || popcnt3(neighbours)
if center==false
	return popcnt3(neighbours)
\end{lstlisting}

Мы можем избавиться от конструкции ``if'':

\begin{lstlisting}
result=(center==true && (popcnt2(neighbours) || popcnt3(neighbours))) || (center==false && popcnt3(neighbours))
\end{lstlisting}

\dots где ``center'' это состояние центральной клетки, ``neighbours'' это 8 соседних клеток, popcnt2 это ф-ция,
возвращающая True если имеет на входе именно 2 бита, popcnt3 это тоже самое, только для 3-х бит
(такие же, как я использовал с своем примере с ``Сапёром'' (\ref{minesweeper_SAT})).

Используя Wolfram Mathematica, я в начале создаю функции-помошники и таблицу истинности для ф-ции, которая будет возвращать
\textit{true}, если ячейка должна присутствовать в следующем состоянии, либо \textit{false}, если нет:

\begin{lstlisting}
In[1]:= popcount[n_Integer]:=IntegerDigits[n,2] // Total

In[2]:= popcount2[n_Integer]:=Equal[popcount[n],2]

In[3]:= popcount3[n_Integer]:=Equal[popcount[n],3]

In[4]:= newcell[center_Integer,neighbours_Integer]:=(center==1 && (popcount2[neighbours]|| popcount3[neighbours]))||
(center==0 && popcount3[neighbours])

In[13]:= NewCellIsTrue=Flatten[Table[Join[{center},PadLeft[IntegerDigits[neighbours,2],8]] ->
Boole[newcell[center, neighbours]],{neighbours,0,255},{center,0,1}]]

Out[13]= {{0,0,0,0,0,0,0,0,0}->0,
{1,0,0,0,0,0,0,0,0}->0,
{0,0,0,0,0,0,0,0,1}->0,
{1,0,0,0,0,0,0,0,1}->0,
{0,0,0,0,0,0,0,1,0}->0,
{1,0,0,0,0,0,0,1,0}->0,
{0,0,0,0,0,0,0,1,1}->0,
{1,0,0,0,0,0,0,1,1}->1,

...

\end{lstlisting}

Теперь создаем \ac{CNF}-выражение из таблицы истинности:

\begin{lstlisting}
In[14]:= BooleanConvert[BooleanFunction[NewCellIsTrue,{center,a,b,c,d,e,f,g,h}],"CNF"]
Out[14]= (!a||!b||!c||!d)&&(!a||!b||!c||!e)&&(!a||!b||!c||!f)&&(!a||!b||!c||!g)&&(!a||!b||!c||!h)&&
(!a||!b||!d||!e)&&(!a||!b||!d||!f)&&(!a||!b||!d||!g)&&(!a||!b||!d||!h)&&(!a||!b||!e||!f)&&
(!a||!b||!e||!g)&&(!a||!b||!e||!h)&&(!a||!b||!f||!g)&&(!a||!b||!f||!h)&&(!a||!b||!g||!h)&&
(!a||!c||!d||!e)&&(!a||!c||!d||!f)&&(!a||!c||!d||!g)&&(!a||!c||!d||!h)&&(!a||!c||!e||!f)&&
(!a||!c||!e||!g)&&(!a||!c||!e||!h)&&(!a||!c||!f||!g)&&(!a||!c||!f||!h)&&

...

\end{lstlisting}

Также, нам нужна вторая ф-ция, \textit{инвертированная}, которая вернет \textit{true},
если ячейка должна отсутствовать в следующем состоянии, или \textit{false} в противном случае:

\begin{lstlisting}
In[15]:= NewCellIsFalse=Flatten[Table[Join[{center},PadLeft[IntegerDigits[neighbours,2],8]] ->
Boole[Not[newcell[center, neighbours]]],{neighbours,0,255},{center,0,1}]]
Out[15]= {{0,0,0,0,0,0,0,0,0}->1,
{1,0,0,0,0,0,0,0,0}->1,
{0,0,0,0,0,0,0,0,1}->1,
{1,0,0,0,0,0,0,0,1}->1,
{0,0,0,0,0,0,0,1,0}->1,

...

In[16]:= BooleanConvert[BooleanFunction[NewCellIsFalse,{center,a,b,c,d,e,f,g,h}],"CNF"]
Out[16]= (!a||!b||!c||d||e||f||g||h)&&(!a||!b||c||!d||e||f||g||h)&&(!a||!b||c||d||!e||f||g||h)&&
(!a||!b||c||d||e||!f||g||h)&&(!a||!b||c||d||e||f||!g||h)&&(!a||!b||c||d||e||f||g||!h)&&
(!a||!b||!center||d||e||f||g||h)&&(!a||b||!c||!d||e||f||g||h)&&(!a||b||!c||d||!e||f||g||h)&&
(!a||b||!c||d||e||!f||g||h)&&(!a||b||!c||d||e||f||!g||h)&&(!a||b||!c||d||e||f||g||!h)&&
(!a||b||c||!d||!e||f||g||h)&&(!a||b||c||!d||e||!f||g||h)&&(!a||b||c||!d||e||f||!g||h)&&

...

\end{lstlisting}

% TODO \ref{}
Используя тот же способ, что и в примере с ``Сапёром'', я могу конвертировать \ac{CNF}-выражение в список клозов:

\begin{lstlisting}
def mathematica_to_CNF (s, center, a):
    s=s.replace("center", center)
    s=s.replace("a", a[0]).replace("b", a[1]).replace("c", a[2]).replace("d", a[3])
    s=s.replace("e", a[4]).replace("f", a[5]).replace("g", a[6]).replace("h", a[7])
    s=s.replace("!", "-").replace("||", " ").replace("(", "").replace(")", "")
    s=s.split ("&&")
    return s
\end{lstlisting}

И снова, как в примере с ``Сапёром'', здесь есть невидимая рамка, чтобы сделать обработку проще.
Переменные в \ac{SAT} нумеруются как и в предыдущем примере:

\begin{lstlisting}
 1    2   3   4   5   6   7   8   9  10  11
12   13  14  15  16  17  18  19  20  21  22
23   24  25  26  27  28  29  30  31  32  33
34   35  36  37  38  39  40  41  42  43  44

...

100 101 102 103 104 105 106 107 108 109 110
111 112 113 114 115 116 117 118 119 120 121
\end{lstlisting}

Также, для упрощения, здесь есть и видимая рамка, всегда равняющаяся \textit{False}.

Теперь работающий исходник.
Всякий раз, когда встречаем ``*'' в \TT{final\_state[]}, мы добавляем клозы сгенерированные ф-цией
\TT{cell\_is\_true()}, или \TT{cell\_is\_false()}, если наоборот.
Когда получаем решение, мы его инвертируем и добавляем в список клозов, так что когда minisat будет исполняться
в следующий раз, он пропустит решение, которое только что было выведено.

\begin{lstlisting}
...

def cell_is_false (center, a):
    s="(!a||!b||!c||d||e||f||g||h)&&(!a||!b||c||!d||e||f||g||h)&&(!a||!b||c||d||!e||f||g||h)&&" \
      "(!a||!b||c||d||e||!f||g||h)&&(!a||!b||c||d||e||f||!g||h)&&(!a||!b||c||d||e||f||g||!h)&&" \
      "(!a||!b||!center||d||e||f||g||h)&&(!a||b||!c||!d||e||f||g||h)&&(!a||b||!c||d||!e||f||g||h)&&" \
      "(!a||b||!c||d||e||!f||g||h)&&(!a||b||!c||d||e||f||!g||h)&&(!a||b||!c||d||e||f||g||!h)&&" \
      "(!a||b||c||!d||!e||f||g||h)&&(!a||b||c||!d||e||!f||g||h)&&(!a||b||c||!d||e||f||!g||h)&&" \
      "(!a||b||c||!d||e||f||g||!h)&&(!a||b||c||d||!e||!f||g||h)&&(!a||b||c||d||!e||f||!g||h)&&" \
      "(!a||b||c||d||!e||f||g||!h)&&(!a||b||c||d||e||!f||!g||h)&&(!a||b||c||d||e||!f||g||!h)&&" \
      "(!a||b||c||d||e||f||!g||!h)&&(!a||!c||!center||d||e||f||g||h)&&(!a||c||!center||!d||e||f||g||h)&&" \
      "(!a||c||!center||d||!e||f||g||h)&&(!a||c||!center||d||e||!f||g||h)&&(!a||c||!center||d||e||f||!g||h)&&" \
      "(!a||c||!center||d||e||f||g||!h)&&(a||!b||!c||!d||e||f||g||h)&&(a||!b||!c||d||!e||f||g||h)&&" \
      "(a||!b||!c||d||e||!f||g||h)&&(a||!b||!c||d||e||f||!g||h)&&(a||!b||!c||d||e||f||g||!h)&&" \
      "(a||!b||c||!d||!e||f||g||h)&&(a||!b||c||!d||e||!f||g||h)&&(a||!b||c||!d||e||f||!g||h)&&" \
      "(a||!b||c||!d||e||f||g||!h)&&(a||!b||c||d||!e||!f||g||h)&&(a||!b||c||d||!e||f||!g||h)&&" \
      "(a||!b||c||d||!e||f||g||!h)&&(a||!b||c||d||e||!f||!g||h)&&(a||!b||c||d||e||!f||g||!h)&&" \
      "(a||!b||c||d||e||f||!g||!h)&&(a||b||!c||!d||!e||f||g||h)&&(a||b||!c||!d||e||!f||g||h)&&" \
      "(a||b||!c||!d||e||f||!g||h)&&(a||b||!c||!d||e||f||g||!h)&&(a||b||!c||d||!e||!f||g||h)&&" \
      "(a||b||!c||d||!e||f||!g||h)&&(a||b||!c||d||!e||f||g||!h)&&(a||b||!c||d||e||!f||!g||h)&&" \
      "(a||b||!c||d||e||!f||g||!h)&&(a||b||!c||d||e||f||!g||!h)&&(a||b||c||!d||!e||!f||g||h)&&" \
      "(a||b||c||!d||!e||f||!g||h)&&(a||b||c||!d||!e||f||g||!h)&&(a||b||c||!d||e||!f||!g||h)&&" \
      "(a||b||c||!d||e||!f||g||!h)&&(a||b||c||!d||e||f||!g||!h)&&(a||b||c||d||!e||!f||!g||h)&&" \
      "(a||b||c||d||!e||!f||g||!h)&&(a||b||c||d||!e||f||!g||!h)&&(a||b||c||d||e||!f||!g||!h)&&" \
      "(!b||!c||!center||d||e||f||g||h)&&(!b||c||!center||!d||e||f||g||h)&&(!b||c||!center||d||!e||f||g||h)&&" \
      "(!b||c||!center||d||e||!f||g||h)&&(!b||c||!center||d||e||f||!g||h)&&(!b||c||!center||d||e||f||g||!h)&&" \
      "(b||!c||!center||!d||e||f||g||h)&&(b||!c||!center||d||!e||f||g||h)&&(b||!c||!center||d||e||!f||g||h)&&" \
      "(b||!c||!center||d||e||f||!g||h)&&(b||!c||!center||d||e||f||g||!h)&&(b||c||!center||!d||!e||f||g||h)&&" \
      "(b||c||!center||!d||e||!f||g||h)&&(b||c||!center||!d||e||f||!g||h)&&(b||c||!center||!d||e||f||g||!h)&&" \
      "(b||c||!center||d||!e||!f||g||h)&&(b||c||!center||d||!e||f||!g||h)&&(b||c||!center||d||!e||f||g||!h)&&" \
      "(b||c||!center||d||e||!f||!g||h)&&(b||c||!center||d||e||!f||g||!h)&&(b||c||!center||d||e||f||!g||!h)"

    return mathematica_to_CNF(s, center, a)

def cell_is_true (center, a):
    s="(!a||!b||!c||!d)&&(!a||!b||!c||!e)&&(!a||!b||!c||!f)&&(!a||!b||!c||!g)&&(!a||!b||!c||!h)&&" \
      "(!a||!b||!d||!e)&&(!a||!b||!d||!f)&&(!a||!b||!d||!g)&&(!a||!b||!d||!h)&&(!a||!b||!e||!f)&&" \
      "(!a||!b||!e||!g)&&(!a||!b||!e||!h)&&(!a||!b||!f||!g)&&(!a||!b||!f||!h)&&(!a||!b||!g||!h)&&" \
      "(!a||!c||!d||!e)&&(!a||!c||!d||!f)&&(!a||!c||!d||!g)&&(!a||!c||!d||!h)&&(!a||!c||!e||!f)&&" \
      "(!a||!c||!e||!g)&&(!a||!c||!e||!h)&&(!a||!c||!f||!g)&&(!a||!c||!f||!h)&&(!a||!c||!g||!h)&&" \
      "(!a||!d||!e||!f)&&(!a||!d||!e||!g)&&(!a||!d||!e||!h)&&(!a||!d||!f||!g)&&(!a||!d||!f||!h)&&" \
      "(!a||!d||!g||!h)&&(!a||!e||!f||!g)&&(!a||!e||!f||!h)&&(!a||!e||!g||!h)&&(!a||!f||!g||!h)&&" \
      "(a||b||c||center||d||e||f)&&(a||b||c||center||d||e||g)&&(a||b||c||center||d||e||h)&&" \
      "(a||b||c||center||d||f||g)&&(a||b||c||center||d||f||h)&&(a||b||c||center||d||g||h)&&" \
      "(a||b||c||center||e||f||g)&&(a||b||c||center||e||f||h)&&(a||b||c||center||e||g||h)&&" \
      "(a||b||c||center||f||g||h)&&(a||b||c||d||e||f||g)&&(a||b||c||d||e||f||h)&&(a||b||c||d||e||g||h)&&" \
      "(a||b||c||d||f||g||h)&&(a||b||c||e||f||g||h)&&(a||b||center||d||e||f||g)&&(a||b||center||d||e||f||h)&&" \
      "(a||b||center||d||e||g||h)&&(a||b||center||d||f||g||h)&&(a||b||center||e||f||g||h)&&" \
      "(a||b||d||e||f||g||h)&&(a||c||center||d||e||f||g)&&(a||c||center||d||e||f||h)&&" \
      "(a||c||center||d||e||g||h)&&(a||c||center||d||f||g||h)&&(a||c||center||e||f||g||h)&&" \
      "(a||c||d||e||f||g||h)&&(a||center||d||e||f||g||h)&&(!b||!c||!d||!e)&&(!b||!c||!d||!f)&&" \
      "(!b||!c||!d||!g)&&(!b||!c||!d||!h)&&(!b||!c||!e||!f)&&(!b||!c||!e||!g)&&(!b||!c||!e||!h)&&" \
      "(!b||!c||!f||!g)&&(!b||!c||!f||!h)&&(!b||!c||!g||!h)&&(!b||!d||!e||!f)&&(!b||!d||!e||!g)&&" \
      "(!b||!d||!e||!h)&&(!b||!d||!f||!g)&&(!b||!d||!f||!h)&&(!b||!d||!g||!h)&&(!b||!e||!f||!g)&&" \
      "(!b||!e||!f||!h)&&(!b||!e||!g||!h)&&(!b||!f||!g||!h)&&(b||c||center||d||e||f||g)&&" \
      "(b||c||center||d||e||f||h)&&(b||c||center||d||e||g||h)&&(b||c||center||d||f||g||h)&&" \
      "(b||c||center||e||f||g||h)&&(b||c||d||e||f||g||h)&&(b||center||d||e||f||g||h)&&" \
      "(!c||!d||!e||!f)&&(!c||!d||!e||!g)&&(!c||!d||!e||!h)&&(!c||!d||!f||!g)&&(!c||!d||!f||!h)&&" \
      "(!c||!d||!g||!h)&&(!c||!e||!f||!g)&&(!c||!e||!f||!h)&&(!c||!e||!g||!h)&&(!c||!f||!g||!h)&&" \
      "(c||center||d||e||f||g||h)&&(!d||!e||!f||!g)&&(!d||!e||!f||!h)&&(!d||!e||!g||!h)&&(!d||!f||!g||!h)&&" \
      "(!e||!f||!g||!h)"

    return mathematica_to_CNF(s, center, a)

...
\end{lstlisting}

( \url{https://github.com/DennisYurichev/SAT_SMT_article/blob/master/SAT/GoL/GoL_SAT_utils.py} )

\lstinputlisting{SAT/GoL/reverse1.py}

( \url{https://github.com/DennisYurichev/SAT_SMT_article/blob/master/SAT/GoL/reverse1.py} )

И вот результат:

\begin{lstlisting}
HEIGHT= 3 WIDTH= 3
2525 clauses
.*.
*.*
.*.
1.rle written

2526 clauses
.**
*..
*.*
2.rle written

2527 clauses
**.
..*
*.*
3.rle written

2528 clauses
*.*
*..
.**
4.rle written

2529 clauses
*.*
..*
**.
5.rle written

2530 clauses
*.*
.*.
*.*
6.rle written

2531 clauses
unsat!
\end{lstlisting}

Первый результат такой же, как и изначальное состояние.
Действительно: это ``натюрморт'', т.е., состояние которое никогда не меняется, и это корректное решение.
Последнее решение также корректно.

Теперь проблема: 2-е, 3-е, 4-е и 5-е решения эквивалентны друг другу, они просто или отражены или повернуты.
На самом деле, это зеркальная\footnote{\url{https://en.wikipedia.org/wiki/Reflection_symmetry}} и 
ротационная\footnote{\url{https://en.wikipedia.org/wiki/Rotational_symmetry}} симметрии.
Мы можем это легко решить: мы будем брать каждое решение, отражать его и крутить, и добавлять в инвертированном виде
в список клозов, так что minisat при следующем запуске их пропустит:

\begin{lstlisting}

...

while True:
    solution=try_again(clauses)
    clauses.append(negate_clause(grid_to_clause(solution, H, W)))
    clauses.append(negate_clause(grid_to_clause(reflect_vertically(solution), H, W)))
    clauses.append(negate_clause(grid_to_clause(reflect_horizontally(solution), H, W)))
    # is this square?
    if W==H:
        clauses.append(negate_clause(grid_to_clause(rotate_square_array(solution,1), H, W)))
        clauses.append(negate_clause(grid_to_clause(rotate_square_array(solution,2), H, W)))
        clauses.append(negate_clause(grid_to_clause(rotate_square_array(solution,3), H, W)))
    print ""

...

\end{lstlisting}

( \url{https://github.com/DennisYurichev/SAT_SMT_article/blob/master/SAT/GoL/reverse2.py} )

Ф-ции \TT{reflect\_vertically()}, \TT{reflect\_horizontally} and \TT{rotate\_squarearray()} просто преобразуют массив.

Теперь у нас всего 3 решения:

\begin{lstlisting}
HEIGHT= 3 WIDTH= 3
2525 clauses
.*.
*.*
.*.
1.rle written

2531 clauses
.**
*..
*.*
2.rle written

2537 clauses
*.*
.*.
*.*
3.rle written

2543 clauses
unsat!
\end{lstlisting}

У этого только один предок:

\begin{lstlisting}
final_state=[
" * ",
" * ",
" * "]
_PRE_END

_PRE_BEGIN
HEIGHT= 3 WIDTH= 3
2503 clauses
...
***
...
1.rle written

2509 clauses
unsat!
\end{lstlisting}

Это осциллятор, конечно.

Как много состояний могут привести к такой картинке?

\begin{lstlisting}
final_state=[
"  *  ",
"     ",
" **  ",
"  *  ",
"  *  ",
" *** "]
\end{lstlisting}

28, вот некоторые:

\begin{lstlisting}
HEIGHT= 6 WIDTH= 5
5217 clauses
.*.*.
..*..
.**.*
..*..
..*.*
.**..
1.rle written

5220 clauses
.*.*.
..*..
.**.*
..*..
*.*.*
.**..
2.rle written

5223 clauses
..*.*
..**.
.**..
....*
*.*.*
.**..
3.rle written

5226 clauses
..*.*
..**.
.**..
*...*
..*.*
.**..
4.rle written

...

\end{lstlisting}

Теперь самая большая, ``space invader'':

\begin{lstlisting}
final_state=[
"             ",
"   *     *   ",
"    *   *    ",
"   *******   ",
"  ** *** **  ",
" *********** ",
" * ******* * ",
" * *     * * ",
"    ** **    ",
"             "]
\end{lstlisting}

\begin{lstlisting}
HEIGHT= 10 WIDTH= 13
16469 clauses
..*.*.**.....
.....*****...
....**..*....
......*...*..
..**...*.*...
.*..*.*.**..*
*....*....*.*
..*.*..*.....
..*.....*.*..
....**..*.*..
1.rle written

16472 clauses
*.*.*.**.....
.....*****...
....**..*....
......*...*..
..**...*.*...
.*..*.*.**..*
*....*....*.*
..*.*..*.....
..*.....*.*..
....**..*.*..
2.rle written

16475 clauses
..*.*.**.....
*....*****...
....**..*....
......*...*..
..**...*.*...
.*..*.*.**..*
*....*....*.*
..*.*..*.....
..*.....*.*..
....**..*.*..
3.rle written

...

\end{lstlisting}

Не знаю, сколько возможных состояний могут привести к ``space invader''-у, вероятно, очень много.
Пришлось остановить.
И во время исполнения, вывод решений постепенно замедляется, потому что количество клозов возрастает
(из-за добавления инвертированных решений).

Все решения также экспортируются в RLE-файлы, которые можно открывать при помощи
Golly\footnote{\url{http://golly.sourceforge.net/}}.

\subsubsection{Поиск ``натюрмортов''}

``Натюрморт'' в терминах игры это состояние, которое вообще не меняется.

В начале, используя предыдущие определения, мы определим таблицу истинности для ф-ции, которая будет возвращать \textit{true},
если центральная клетка следующего состояния такая же, как она и была в предыдущем состоянии, т.е., она не менялась.

\begin{lstlisting}
In[17]:= stillife=Flatten[Table[Join[{center},PadLeft[IntegerDigits[neighbours,2],8]]->
Boole[Boole[newcell[center,neighbours]]==center],{neighbours,0,255},{center,0,1}]]
Out[17]= {{0,0,0,0,0,0,0,0,0}->1,
{1,0,0,0,0,0,0,0,0}->0,
{0,0,0,0,0,0,0,0,1}->1,
{1,0,0,0,0,0,0,0,1}->0,

...

In[18]:= BooleanConvert[BooleanFunction[stillife,{center,a,b,c,d,e,f,g,h}],"CNF"]
Out[18]= (!a||!b||!c||!center||!d)&&(!a||!b||!c||!center||!e)&&(!a||!b||!c||!center||!f)&&
(!a||!b||!c||!center||!g)&&(!a||!b||!c||!center||!h)&&(!a||!b||!c||center||d||e||f||g||h)&&
(!a||!b||c||center||!d||e||f||g||h)&&(!a||!b||c||center||d||!e||f||g||h)&&(!a||!b||c||center||d||e||!f||g||h)&&
(!a||!b||c||center||d||e||f||!g||h)&&(!a||!b||c||center||d||e||f||g||!h)&&(!a||!b||!center||!d||!e)&&

...

\end{lstlisting}

\lstinputlisting{SAT/GoL/SL1.py}

( \url{https://github.com/DennisYurichev/SAT_SMT_article/blob/master/SAT/GoL/SL1.py} )

Что получим для $2 \cdot 2$?

\begin{lstlisting}
1881 clauses
..
..
1.rle written

1887 clauses
**
**
2.rle written

1893 clauses
unsat!
\end{lstlisting}

Оба решения корректны: пустой квадрат трансформируется в пустой квадрат (клетки не будут рождены).
Квадрат $2 \cdot 2$ также известен как ''натюрморт''.

Что насчет квадрата $3 \cdot 3$?

\begin{lstlisting}
2887 clauses
...
...
...
1.rle written

2893 clauses
.**
.**
...
2.rle written

2899 clauses
.**
*.*
**.
3.rle written

2905 clauses
.*.
*.*
**.
4.rle written

2911 clauses
.*.
*.*
.*.
5.rle written

2917 clauses
unsat!
\end{lstlisting}

Вот проблема: мы видим знакомый квадрат $2 \cdot 2$, но он сдвинут.
Это корректное решение, но нам оно не интересно, потому что мы его уже видели.

Что мы можем, это добавить еще одно условие. Мы можем заставить minisat искать решения без пустых рядов и столбцов.
Это легко.
Вот переменные в SAT для квадрата $5 \cdot 5$:

\begin{lstlisting}
1   2  3  4  5
6   7  8  9 10
11 12 13 14 15
16 17 18 19 20
21 22 23 24 25
\end{lstlisting}

Каждый клоз это клоз ``ИЛИ'', так что, всё что нужно, это добавить 5 клозов:

\begin{lstlisting}
1 OR 2 OR 3 OR 4 OR 5
6 OR 7 OR 8 OR 9 OR 10

...

\end{lstlisting}

Жто значит, что каждый ряд должен где-то иметь минимум одно значение \textit{True}.
Мы можем это делать и для каждого столбца.

\begin{lstlisting}

...

    # each row must contain at least one cell!
    for r in range(H):
        clauses.append(" ".join([coords_to_var(r, c, H, W) for c in range(W)]))

    # each column must contain at least one cell!
    for c in range(W):
        clauses.append(" ".join([coords_to_var(r, c, H, W) for r in range(H)]))

...

\end{lstlisting}

( \url{https://github.com/DennisYurichev/SAT_SMT_article/blob/master/SAT/GoL/SL2.py} )

Теперь мы видим, что квадрат $3 \cdot 3$ имеет 3 возможных натюрморта:

\begin{lstlisting}
2893 clauses
.*.
*.*
**.
1.rle written

2899 clauses
.*.
*.*
.*.
2.rle written

2905 clauses
.**
*.*
**.
3.rle written

2911 clauses
unsat!
\end{lstlisting}

$4 \cdot 4$ имеет 7:

\begin{lstlisting}
4169 clauses
..**
...*
***.
*...
1.rle written

4175 clauses
..**
..*.
*.*.
**..
2.rle written

4181 clauses
..**
.*.*
*.*.
**..
3.rle written

4187 clauses
..*.
.*.*
*.*.
**..
4.rle written

4193 clauses
.**.
*..*
*.*.
.*..
5.rle written

4199 clauses
..*.
.*.*
*.*.
.*..
6.rle written

4205 clauses
.**.
*..*
*..*
.**.
7.rle written

4211 clauses
unsat!
\end{lstlisting}

Когда я пробую большие квадраты, например $20 \cdot 20$, происходит смешное.
Прежде всего, minisat находит решение, не очень красивое эстетически, но всё же корректное, как:

\begin{lstlisting}
61033 clauses
....**.**.**.**.**.*
**..*.**.**.**.**.**
*...................
.*..................
**..................
*...................
.*..................
**..................
*...................
.*..................
**..................
*...................
.*..................
**..................
*...................
.*..................
..*.................
...*................
***.................
*...................
1.rle written

...

\end{lstlisting}

Действительно: все ряды и столбцы имеют миниму одно значение \textit{True}.

Затем minisat начинает добавлять маленькие натюрморты в общую картину:

\begin{lstlisting}
61285 clauses
.**....**...**...**.
.**...*..*.*.*...*..
.......**...*......*
..................**
...**............*..
...*.*...........*..
....*.*........**...
**...*.*...**..*....
*.*...*....*....*...
.*..........****.*..
................*...
..*...**..******....
.*.*..*..*..........
*..*...*.*..****....
***.***..*.*....*...
....*..***.**..**...
**.*..*.............
.*.**.**..........**
*..*..*..*......*..*
**..**..**......**..
43.rle written
\end{lstlisting}

Другими словами, результат это квадрат, состоящий из меньших натюрмортов.
Он затем немного меняет эти части, двигает их вперед и назад.
Это жульничество?
Так или иначе, он делает это в соответствии с четкими правилами, которые мы определили.

Но мы хотим более \textit{плотную} картину. Мы можем добавить правило: во всех кусках по 5 клеток должна быть
минимум одна \textit{True}-клетка.
Чтобы добиться этого, мы просто делим весь квадрат на куски по 5 клеток и добавляем клоз для каждого:

\begin{lstlisting}

...

    # make result denser:
    lst=[]
    for r in range(H):
        for c in range(W):
            lst.append(coords_to_var(r, c, H, W))
    # divide them all by chunks and add to clauses:
    CHUNK_LEN=5
    for c in list_partition(lst,len(lst)/CHUNK_LEN):
        tmp=" ".join(c)
        clauses.append(tmp)

...

\end{lstlisting}

( \url{https://github.com/DennisYurichev/SAT_SMT_article/blob/master/SAT/GoL/SL3.py} )

Это действительно поплотнее:

\begin{lstlisting}
61113 clauses
..**.**......*.*.*..
...*.*.....***.**.*.
...*..*...*.......*.
....*.*..*.*......**
...**.*.*..*...**.*.
..*...*.***.....*.*.
...*.*.*......*..*..
****.*..*....*.**...
*....**.*....*.*....
...**..*...**..*....
..*..*....*....*.**.
.*.*.**....****.*..*
..*.*....*.*..*..**.
....*.****..*..*.*..
....**....*.*.**..*.
*.**...****.*..*.**.
**...**.....**.*....
...**..*..**..*.**.*
***.*.*..*.*..*.*.**
*....*....*....*....
1.rle written

61119 clauses
..**.**......*.*.*..
...*.*.....***.**.*.
...*..*...*.......*.
....*.*..*.*......**
...**.*.*..*...**.*.
..*...*.***.....*.*.
...*.*.*......*..*..
****.*..*....*.**...
*....**.*....*.*....
...**..*...**..*....
..*..*....*....*.**.
.*.*.**....****.*..*
..*.*....*.*..*..**.
....*.****..*..*.*..
....**....*.*.**..*.
*.**...****.*..*.**.
**...**.....**.*....
...**..*.***..*.**.*
***.*..*.*..*.*.*.**
*.......*..**.**....
2.rle written

...

\end{lstlisting}

Попробуем еще плотнее, одна обязательная \textit{true}-клетка для каждого 4-клеточного куска:

\begin{lstlisting}
61133 clauses
.**.**...*....**..**
*.*.*...*.*..*..*..*
*....*...*.*..*.**..
.***.*.....*.**.*...
..*.*.....**...*..*.
*......**..*...*.**.
**.....*...*.**.*...
...**...*...**..*...
**.*..*.*......*...*
.*...**.**..***.****
.*....*.*..*..*.*...
**.***...*.**...*.**
.*.*..****.....*..*.
*....*.....**..**.*.
*.***.*..**.*.....**
.*...*..*......**...
...*.*.**......*.***
..**.*.....**......*
*..*.*.**..*.*..***.
**....*.*...*...*...
1.rle written

61139 clauses
.**.**...*....**..**
*.*.*...*.*..*..*..*
*....*...*.*..*.**..
.***.*.....*.**.*...
..*.*.....**...*..*.
*......**..*...*.**.
**.....*...*.**.*...
...**...*...**..*...
**.*..*.*......*...*
.*...**.**..***.****
.*....*.*..*..*.*...
**.***...*.**...*.**
.*.*..****.....*..*.
*....*.....**..**.*.
*.***.*..**.*.....**
.*...*..*......**..*
...*.*.**......*.**.
..**.*.....**....*..
*..*.*.**..*.*...*.*
**....*.*...*.....**
2.rle written

...

\end{lstlisting}

\dots и даже больше: одна клетка на каждый 3-клеточный кусок:

\begin{lstlisting}
61166 clauses
**.*..**...**.**....
*.**..*.*...*.*.*.**
....**..*...*...*.*.
.**..*.*.**.*.*.*.*.
..**.*.*...*.**.*.**
*...*.*.**.*....*.*.
**.*..*...*.*.***..*
.*.*.*.***..**...**.
.*.*.*.*..**...*.*..
**.**.*..*...**.*..*
..*...*.**.**.*.*.**
..*.**.*..*.*.*.*...
**.*.*...*..*.*.*...
.*.*...*.**..*..***.
.*..****.*....**...*
..*.*...*..*...*..*.
.**...*.*.**...*.*..
..*..**.*.*...**.**.
..*.*..*..*..*..*..*
.**.**....**..**..**
1.rle written

61172 clauses
**.*..**...**.**....
*.**..*.*...*.*.*.**
....**..*...*...*.*.
.**..*.*.**.*.*.*.*.
..**.*.*...*.**.*.**
*...*.*.**.*....*.*.
**.*..*...*.*.***..*
.*.*.*.***..**...**.
.*.*.*.*..**...*.*..
**.**.*..*...**.*..*
..*...*.**.**.*.*.**
..*.**.*..*.*.*.*...
**.*.*...*..*.*.*...
.*.*...*.**..*..***.
.*..****.*....**...*
..*.*...*..*...*..*.
.**..**.*.**...*.*..
*..*.*..*.*...**.**.
*..*.*.*..*..*..*..*
.**...*...**..**..**
2.rle written

...

\end{lstlisting}

Это самое плотное. К несчастью, невозможно создать натюрморт с одной обязательной 
\textit{true}-клеткой в каждом 2-клеточном куске.

\subsubsection{Исходный код}

Исходный код и файл для Wolfram Mathematica: \url{https://github.com/DennisYurichev/SAT_SMT_article/tree/master/SAT/GoL}.

