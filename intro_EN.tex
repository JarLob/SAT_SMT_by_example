\section{Latest versions}

Latest version is always available at \url{http://yurichev.com/writings/SAT_SMT_draft-EN.pdf}.
Russian version is at \url{http://yurichev.com/writings/SAT_SMT_draft-RU.pdf}.

New parts are appearing here from time to time, see: \url{https://github.com/DennisYurichev/SAT_SMT_article/blob/master/ChangeLog}.

\section{Thanks}

Armin Biere\footnote{\url{http://fmv.jku.at/biere/}} has patiently answered to my boring questions.

Leonardo Mendonça de Moura\footnote{\url{https://www.microsoft.com/en-us/research/people/leonardo/}},
Nikolaj Bjørner\footnote{\url{https://www.microsoft.com/en-us/research/people/nbjorner/}}
and Mate Soos\footnote{\url{https://www.msoos.org/}} have also helped.

Alex ``clayrat'' Gryzlov has found a bug.

\section{Praise}

``An excellent source of well-worked through and motivating examples of using Z3's python interface.''
\footnote{\url{https://github.com/Z3Prover/z3/wiki}}
(Nikolaj Bjorner, one of Z3's authors).

``Impressive collection of fun examples!''
(Pascal Fontaine\footnote{\url{https://members.loria.fr/PFontaine/}}, one of veriT solver's authors.)

\section{Introduction}

\ac{SAT}/\ac{SMT} solvers can be viewed as solvers of huge systems of equations.
The difference is that \ac{SMT} solvers takes systems in arbitrary format,
while \ac{SAT} solvers are limited to boolean equations in \ac{CNF} form.

A lot of real world problems can be represented as problems of solving system of equations.

\section{Is it a hype? Yet another fad?}

Some people say, this is just another hype.
No, \ac{SAT} is old enough and fundamental to \ac{CS}.
The reason of increased interest to it is that computers gets faster over the last couple decades,
so there are attempts to solve old problems using \ac{SAT}/\ac{SMT}, which were inaccessible in past.

