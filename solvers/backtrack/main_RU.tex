\subsection{Простейший SAT-солвер в \textasciitilde{}120 строках}
\label{SAT_backtrack}

Это простейший SAT-солвер работающий на базе поиска с возвратом (\textit{backtracking}) (не \ac{DPLL}), написанный
на Питоне.
Он использует тот же поиск с возвратом, который можно найти в простейших солверах Судоку и задачи о восьми ферзях.
Он работает значительно медленнее, но, из-за предельной простоты, он также может подсчитывать количество решений.
Например, он может подсчитать все решения для задачи о восьми ферзях (\ref{EightQueens}).

Также, имеется 70 решений для ф-ции POPCNT4
\footnote{\url{https://github.com/DennisYurichev/SAT_SMT_article/blob/master/SAT/backtrack/POPCNT4.cnf}}
(ф-ция истинна, если любые из её 4-х входов из 8-и истинны):

\begin{lstlisting}
SAT
-1 -2 -3 -4 5 6 7 8 0
SAT
-1 -2 -3 4 -5 6 7 8 0
SAT
-1 -2 -3 4 5 -6 7 8 0
SAT
-1 -2 -3 4 5 6 -7 8 0
...

SAT
1 2 3 -4 -5 6 -7 -8 0
SAT
1 2 3 -4 5 -6 -7 -8 0
SAT
1 2 3 4 -5 -6 -7 -8 0
UNSAT
solutions= 70
\end{lstlisting}

Солвер также тестировался на моем взломщике Сапёра основанном на SAT (\ref{minesweeper_SAT}),
и заканчивает работу в разумное время (хотя и медленнее чем MiniSat раз в \textasciitilde{}10).

На б\'{о}льших \ac{CNF}-задачах он зависает.

Исходный код:
% TODO: translate to RU:
\lstinputlisting{SAT/backtrack/SAT_backtrack.py}

Как вы видите, всё что он делает, это перечисляет все возможные решения, но отсекает поисковое дерево настолько рано,
насколько это возможно.
Это и есть поиск с возвратом (\textit{backtracking}).

Файлы: \url{https://github.com/DennisYurichev/SAT_SMT_article/tree/master/SAT/backtrack}.

Некоторые комментарии: \url{https://www.reddit.com/r/compsci/comments/6jn3th/simplest_sat_solver_in_120_lines/}.

