\subsubsection{Oculus VR Flicks and GCD}

I've found this:

\begin{lstlisting}
A flick (frame-tick) is a very small unit of time. It is 1/705600000 of a second, exactly.

1 flick = 1/705600000 second

This unit of time is the smallest time unit which is LARGER than a nanosecond,
and can in integer quantities exactly represent a single frame duration for 
24hz, 25hz, 30hz, 48hz, 50hz, 60hz, 90hz, 100hz, 120hz, and also 1/1000 divisions of each.
This makes it suitable for use via std::chrono::duration and std::ratio 
for doing timing work against the system high resolution clock, which is in nanoseconds,
but doesn't get slightly out of sync when doing common frame rates.

In order to accomodate media playback, we also support some common audio sample rates as well.
This list is not exhaustive, but covers the majority of digital audio formats.
They are 8kHz, 16kHz, 22.05kHz, 24kHz, 32kHz, 44.1kHz, 48kHz, 88.2kHz, 96kHz, and 192kHz. 
While humans can't hear higher than 48kHz, the higher sample rates are used 
for working audio files which might later be resampled or retimed.

The NTSC variations (~29.97, etc) are actually defined as 24 * 1000/1001 and 
30 * 1000/1001, which are impossible to represent exactly in a way where 1 second is exact,
so we don't bother - they'll be inexact in any circumstance.

Details

    1/24 fps frame: 29400000 flicks
    1/25 fps frame: 28224000 flicks
    1/30 fps frame: 23520000 flicks
    1/48 fps frame: 14700000 flicks
    1/50 fps frame: 14112000 flicks
    1/60 fps frame: 11760000 flicks
    1/90 fps frame: 7840000 flicks
    1/100 fps frame: 7056000 flicks
    1/120 fps frame: 5880000 flicks
    1/8000 fps frame: 88200 flicks
    1/16000 fps frame: 44100 flicks
    1/22050 fps frame: 32000 flicks
    1/24000 fps frame: 29400 flicks
    1/32000 fps frame: 22050 flicks
    1/44100 fps frame: 16000 flicks
    1/48000 fps frame: 14700 flicks
    1/88200 fps frame: 8000 flicks
    1/96000 fps frame: 7350 flicks
    1/192000 fps frame: 3675 flicks
\end{lstlisting}

( \url{https://github.com/OculusVR/Flicks} )

Where the number came from?
Let's enumerate all possible time intervals they want to use and find GCD using Mathematica:

\begin{lstlisting}
In[]:= GCD[1/24, 1/24000, 1/25, 1/25000, 1/30, 1/30000, 1/48, 1/50, 
 1/50000, 1/60, 1/60000, 1/90, 1/90000, 1/100, 1/100000, 1/120, 
 1/120000, 1/8000, 1/16000, 1/22050, 1/24000, 1/32000, 1/44100, 
 1/48000, 1/88200, 1/96000, 1/192000]

Out[]= 1/705600000
\end{lstlisting}

Rationale: you may want to play a video with $\frac{1}{50}$ fps and, simultaneously, play audio with 96kHz sampling rate.
Given that, you can change video frame after each 14112000 flicks and change one audio sample after each 7350 flicks.
Use any other video fps and any audio sampling rate and you will have all time periods as integer numbers.
No ratios any more.

On contrary, one nanosecond wouldn't fit: try to represent 1/30 second in nanoseconds, this is (still) ratio: 33333.33333... nanoseconds.

