\subsubsection{Explanation of the \ac{GCD}}
\label{GCD}

What is Greatest common divisor (\ac{GCD})?

Let's suppose, you want to cut a rectangle by squares. What is maximal square could be?

For a 14*8 rectangle, this is 2*2 square:

% TODO make these blocks horizontally arranged:
\begin{lstlisting}
**************
**************
**************
**************
**************
**************
**************
**************
\end{lstlisting}

% TODO tex symbol:
->

\begin{lstlisting}
** ** ** ** ** ** **
** ** ** ** ** ** **
                  
** ** ** ** ** ** **
** ** ** ** ** ** **
                  
** ** ** ** ** ** **
** ** ** ** ** ** **
                  
** ** ** ** ** ** **
** ** ** ** ** ** **
\end{lstlisting}

What for 14*7 rectangle? It's 7*7 square:

\begin{lstlisting}
**************
**************
**************
**************
**************
**************
**************
\end{lstlisting}

->

\begin{lstlisting}
******* *******
******* *******
******* *******
******* *******
******* *******
******* *******
******* *******
\end{lstlisting}

14*9 rectangle? 1, i.e., smallest possible.

\begin{lstlisting}
**************
**************
**************
**************
**************
**************
**************
**************
**************
**************
\end{lstlisting}

Also, GCD can be used to determine biggest ``granule'' for problem like \ref{XkcdILP}, and this is 0.05 or 5 for that example.

To compute GCD, one of the oldest algorithms is used: \href{https://en.wikipedia.org/wiki/Euclidean_algorithm}{Euclidean algorithm}.
But, I can demonstrate how to make things much less efficient, but more spectacular.

To find GCD of 14 and 8, we are going to solve this system of equations:

% TODO texify
\begin{lstlisting}
x*GCD=14
y*GCD=8
\end{lstlisting}

Then we drop $x$ and $y$, we don't need them.
This system can be solved using paper and pencil, but GCD must be as big as possible.
Here we can use Z3 in MaxSMT mode:

\begin{lstlisting}
#!/usr/bin/env python

from z3 import *

opt = Optimize()

x,y,GCD=Ints('x y GCD')

opt.add(x*GCD==14)
opt.add(y*GCD==8)

h=opt.maximize(GCD)

print (opt.check())
print (opt.model())
\end{lstlisting}

That works:

\begin{lstlisting}
sat
[y = 4, x = 7, GCD = 2]
\end{lstlisting}

What if we need to find GCD for 3 numbers?
Maybe we are going to fill a space with biggest possible cubes?

\begin{lstlisting}
#!/usr/bin/env python

from z3 import *

opt = Optimize()

x,y,z,GCD=Ints('x y z GCD')

opt.add(x*GCD==300)
opt.add(y*GCD==333)
opt.add(z*GCD==900)

h=opt.maximize(GCD)

print (opt.check())
print (opt.model())
\end{lstlisting}

This is 3:

\begin{lstlisting}
sat
[z = 300, y = 111, x = 100, GCD = 3]
\end{lstlisting}

In SMT-LIB form:

\lstinputlisting[style=customsmt]{GCD_BV2.smt}

