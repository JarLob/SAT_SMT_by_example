\subsection{Создание наименьшего набора тестов используя Z3}
\label{set_cover}

Однажды я работал над переписыванием большого кода на чистый Си, и были тесты, несколько тысяч.
Процесс тестирования был очень медленный, так что я думал о том, можно ли как-то уменьшить объем тестов.

Мы можем вот что: запустить каждый тест и получить покрытие кода
(информация о том, какие строки кода исполнялись, а какие --- нет).
Затем задача это создать такой набор тестов, при котором покрытие будет максимальным, а кол-во тестов --- минимальным.

На самом деле, это \textit{задача о покрытии множества} (\textit{set cover problem})
(так же известна как \textit{hitting set problem}).
Существуют алгоритмы и попроще (см. в Wikipedia\footnote{\url{https://en.wikipedia.org/wiki/Set_cover_problem}}),
но это можно решить при помощи SMT-солвера.

В начале, для примера я взял код для компрессии/декомпрессии \ac{LZSS}
\footnote{\url{https://github.com/opensource-apple/kext_tools/blob/master/compression.c}},
из исходных годов от Apple.
Такие процедуры тестировать не легко.
Вот моя версия:
\url{https://github.com/DennisYurichev/SAT_SMT_article/blob/master/SMT/set_cover/compression.c}.
Я добавил генерацию случайных входных данных для сжатия.
Эта генерация зависит от некоторого изначального входного значения (\textit{seed}).
Стандартные ф-ции \TT{srand()}/\TT{rand()} не рекомендуется использовать, но для такой простой задачи как у нас, это ОК.
Я сгенерирую\footnote{\url{https://github.com/DennisYurichev/yurichev.com/blob/master/blog/set_cover/gen_gcov_tests.sh}}
1000 тестов с сидами 0..999, это создаст случайные данны для сжатия/разжатия/проверки.

После того, как процедура сжатия/разжатия закончила работу,
запускается утилита GNU gcov, которая выдает результат вроде:

\begin{lstlisting}
...
     3395:  189:        for (i = 1; i < F; i++) {
     3395:  190:            if ((cmp = key[i] - sp->text_buf[p + i]) != 0)
     2565:  191:                break;
        -:  192:        }
     2565:  193:        if (i > sp->match_length) {
     1291:  194:            sp->match_position = p;
     1291:  195:            if ((sp->match_length = i) >= F)
    #####:  196:                break;
        -:  197:        }
     2565:  198:    }
    #####:  199:    sp->parent[r] = sp->parent[p];
    #####:  200:    sp->lchild[r] = sp->lchild[p];
    #####:  201:    sp->rchild[r] = sp->rchild[p];
    #####:  202:    sp->parent[sp->lchild[p]] = r;
    #####:  203:    sp->parent[sp->rchild[p]] = r;
    #####:  204:    if (sp->rchild[sp->parent[p]] == p)
    #####:  205:        sp->rchild[sp->parent[p]] = r;
...
\end{lstlisting}

Число слева это кол-во, сколько раз исполнилась каждая строка.
\TT{\#\#\#\#\#} означает, что строка кода не была исполнена вовсе.
Второй столбец это номер строки.

Теперь скрипт на Z3Py, который парсит все эти 1000 результатов gcov и находит минимальный \textit{hitting set}:

% TODO: translate:
\lstinputlisting{MaxSMT/set_cover/set_cover.py}

Вот что он выдает (\textasciitilde{}19s на моем старом Intel Quad-Core Xeon E3-1220 3.10GHz):

\begin{lstlisting}
% time python set_cover.py
sat
test_7
test_48
test_134
python set_cover.py  18.95s user 0.03s system 99% cpu 18.988 total
\end{lstlisting}

Нам нужно всего 3 теста, чтобы исполнить (почти) все строки в коде:
это впечатляет, учитывая тот факт, что выбрать эти тесты вручную было бы невероятно трудно!
Результат легко проверить, и снова, при помощи утилиты gcov.

Это иногда называется MaxSAT/MaxSMT --- проблема найти решение, но такое решение,
где некоторая переменная/выражение будет максимальным, насколько это возможно, либо минимальным.

Так же, этот код выдает неверные результаты на Z3 4.4.1, но корректно работает на Z3 4.5.0 (так что, пожалуйста,
обновите).
Это относительно свежая возможность в Z3, так что, вероятно, она не была стабильна в предыдущих версиях?

Файлы: \url{https://github.com/DennisYurichev/SAT_SMT_article/tree/master/SMT/set_cover}.

Дальнейшее чтение:
\url{https://en.wikipedia.org/wiki/Set_cover_problem},
\url{https://en.wikipedia.org/wiki/Maximum_satisfiability_problem},
\url{https://en.wikipedia.org/wiki/Optimization_problem}.

